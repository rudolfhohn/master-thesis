% !TEX root = ../main.tex
% Chapter Introduction

\chapter{Introduction to the thesis topic} % Main chapter title

\label{Chapter1} % Change X to a consecutive number; for referencing this chapter elsewhere, use \ref{ChapterX}

The goal of this thesis is to analyze the different possibilities of constructing conversational agents (chatbots) and what is pragmatically possible today in terms of business applications and what are the next big problems research has to address. Chatbots created a certain interest in the media and in the industry by bringing a new way for businesses to communicate with their customers. Although chatbots exist since 1970s, only recently engineers were able to provide a good enough user experience and use the chatbots as an interaction channel between the customer and the company. Being robust, understanding and reliable are the keys to attract customers using chatbots.

We also see new forms of User Experience requirements. Ergonoms and information visualisation specialists worked these past 20 years to provide best practices for desktop, web and mobile applications to let people enjoy the use of the application. However, with chatbots the main interaction relies on direct communication which rises new requirements.
Character development and character management are new challenges and new jobs that rise with new chatbots~\citep{1704.04579}.
Script writers from movie and theaters industries can bring their knowledge to create and imagine the ``correct'' way of discussing with a customer depending on its age, its culture or its gender.

% The model architecture baseline uses a machine translation model architecture. However, instead of translating from a language to another (e.g. French to English), the model ``translates'' a sentence to another sentence and overall creates a conversational agent, a machine with discussing capabilities.

Chatbots are everyday present in our lives. Apple, Amazon, Google all have their advanced chatbots with different capabilities. For example, an iOS user can ask Siri to add a reminder, send a message or even, when coupled with domotics, turn off lights. Other businesses create chatbots to improve customer experience when selling services or products or booking a hotel or a flight. Facebook Messenger have created a developer-friendly interface to help businesses and people create a chatbot and enable it for all the 1.37 billion daily active users Facebook has~\citep{facebook3rdquarter2017}. Dialogflow is an application that manages the backend of a chatbot. A developer uses their services to create scenarios and prepare the answers, the application will take care of all the natural language processing (NLP) and provide an intelligent conversational agent.

These commercial applications are possible thanks to the research done in both NLP and deep learning fields in recent years. The two main approaches of constructing a chatbot are the rule-based approach and the generative approach. The first one exists for more than 20 years and has limitations. Developers write down all the possible sentences the chabot can encounter and provide the related answer. The first main issue rises when the user does not use the exact same vocabulary or write a sentence differently from what the developer planned and the chatbot is no longer able to answer.
One of the technic used for these type of chatbot is known as the Artificial Intelligence Markup Language (AIML). As a markup language, developers write questions and the adequate answers in a file read by the AIML engine to interact with the user.

The generative approach has appeared recently and take advantage of the progress in Machine Learning (ML) and more particularly of Deep Neural Networks (DNN). The idea of the generative models uses the underlying idea of any ML models ; given a large dataset, the model will be able to learn from the data and predict a value for any input.
Basically, when a user sends a text to a chatbot, the text becomes the input of the ML model which generates an answer based on what it learned, and the chatbot responds to the user. The main advantage, which also answers the main issue of the rule-based approach, is that the model does not need the exact intended text, it can understand synonyms, different sentence constructions and still understand the intent of the user. However, the main disadvantage is the quantity of data needed to train a DNN model. Millions of examples are needed to train a robust generative model.

This thesis is done for a startup based at EPFL that innovates in the way managers interact with and manage their teams.
Understanding what is today possible to do with chatbots is important to help the startup refine its value proposition and build a roadmap for what will be possible tomorrow.
