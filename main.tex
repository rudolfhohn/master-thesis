%%%%%%%%%%%%%%%%%%%%%%%%%%%%%%%%%%%%%%%%%
% Masters/Doctoral Thesis
% LaTeX Template
% Version 2.4 (22/11/16)
%
% This template has been downloaded from:
% http://www.LaTeXTemplates.com
%
% Version 2.x major modifications by:
% Vel (vel@latextemplates.com)
%
% This template is based on a template by:
% Steve Gunn (http://users.ecs.soton.ac.uk/srg/softwaretools/document/templates/)
% Sunil Patel (http://www.sunilpatel.co.uk/thesis-template/)
%
% Template license:
% CC BY-NC-SA 3.0 (http://creativecommons.org/licenses/by-nc-sa/3.0/)
%
%%%%%%%%%%%%%%%%%%%%%%%%%%%%%%%%%%%%%%%%%

%----------------------------------------------------------------------------------------
%	PACKAGES AND OTHER DOCUMENT CONFIGURATIONS
%----------------------------------------------------------------------------------------

\documentclass[
11pt, % The default document font size, options: 10pt, 11pt, 12pt
% oneside, % Two side (alternating margins) for binding by default, uncomment to switch to one side
english, % ngerman for German
% onehalfspacing, % Single line spacing, alternatives: onehalfspacing or doublespacing
singlespacing, % Single line spacing, alternatives: onehalfspacing or doublespacing
%draft, % Uncomment to enable draft mode (no pictures, no links, overfull hboxes indicated)
%nolistspacing, % If the document is onehalfspacing or doublespacing, uncomment this to set spacing in lists to single
liststotoc, % Uncomment to add the list of figures/tables/etc to the table of contents
%toctotoc, % Uncomment to add the main table of contents to the table of contents
parskip, % Uncomment to add space between paragraphs
% nohyperref, % Uncomment to not load the hyperref package
headsepline, % Uncomment to get a line under the header
%chapterinoneline, % Uncomment to place the chapter title next to the number on one line
%consistentlayout, % Uncomment to change the layout of the declaration, abstract and acknowledgements pages to match the default layout
]{MastersDoctoralThesis} % The class file specifying the document structure

\usepackage[utf8]{inputenc} % Required for inputting international characters
\usepackage[T1]{fontenc} % Output font encoding for international characters

\usepackage{palatino} % Use the Palatino font by default

\usepackage[backend=bibtex,style=authoryear,natbib=true]{biblatex} % Use the bibtex backend with the authoryear citation style (which resembles APA)
\setlength\bibitemsep{.75\baselineskip}

\addbibresource{references.bib} % The filename of the bibliography

\usepackage[autostyle=true]{csquotes} % Required to generate language-dependent quotes in the bibliography


\usepackage{xargs}                      % Use more than one optional parameter in a new commands
%
\usepackage[colorinlistoftodos,prependcaption,textsize=tiny,obeyFinal]{todonotes}
\newcommandx{\unsure}[2][1=]{\todo[linecolor=red,backgroundcolor=red!25,bordercolor=red,#1]{#2}}
\newcommandx{\change}[2][1=]{\todo[linecolor=blue,backgroundcolor=blue!25,bordercolor=blue,#1]{#2}}
\newcommandx{\info}[2][1=]{\todo[linecolor=OliveGreen,backgroundcolor=OliveGreen!25,bordercolor=OliveGreen,#1]{#2}}
\newcommandx{\improvement}[2][1=]{\todo[linecolor=Plum,backgroundcolor=Plum!25,bordercolor=Plum,#1]{#2}}
\newcommandx{\thiswillnotshow}[2][1=]{\todo[disable,#1]{#2}}

\usepackage{amsmath}
\usepackage{amssymb}
\usepackage{bm}
\usepackage[overload]{empheq}

\usepackage{listings}

\usepackage{tabularx}
\usepackage{makecell,multirow}
\usepackage{subcaption}
\usepackage{float}

\usepackage{pdflscape}
\usepackage{longtable}
% \usepackage[table]{xcolor}

\usepackage{booktabs}

\newcommand{\tabhead}[1]{\textbf{#1}}

\definecolor{codegray}{gray}{0.9}
\newcommand{\code}[1]{{\texttt{#1}}}
\newcommand{\codeb}[1]{{\textbf{\texttt{#1}}}}
\newcommand{\codeg}[1]{\colorbox{codegray}{\texttt{#1}}}

% siunitx settings
\sisetup{round-mode=places,round-precision=2}

% \usepackage{subfloats}

%----------------------------------------------------------------------------------------
%	MARGIN SETTINGS
%----------------------------------------------------------------------------------------

\geometry{
	paper=a4paper, % Change to letterpaper for US letter
	inner=2.5cm, % Inner margin
	outer=3.8cm, % Outer margin
	bindingoffset=.5cm, % Binding offset
	top=1.5cm, % Top margin
	bottom=1.5cm, % Bottom margin
	% showframe, % Uncomment to show how the type block is set on the page
}

%----------------------------------------------------------------------------------------
%	THESIS INFORMATION
%----------------------------------------------------------------------------------------

\thesistitle{Exploration of Artificial Intelligence Technics for the Construction of Conversational Agents} % Your thesis title, this is used in the title and abstract, print it elsewhere with \ttitle
\supervisor{Dr. Guido \textsc{Bologna}} % Your supervisor's name, this is used in the title page, print it elsewhere with \supname
\examiner{} % Your examiner's name, this is not currently used anywhere in the template, print it elsewhere with \examname
\degree{Master of Science HES-SO in Engineering} % Your degree name, this is used in the title page and abstract, print it elsewhere with \degreename
\author{Rudolf \textsc{Höhn}} % Your name, this is used in the title page and abstract, print it elsewhere with \authorname
\addresses{} % Your address, this is not currently used anywhere in the template, print it elsewhere with \addressname

\subject{Computer Science} % Your subject area, this is not currently used anywhere in the template, print it elsewhere with \subjectname
\keywords{} % Keywords for your thesis, this is not currently used anywhere in the template, print it elsewhere with \keywordnames
\university{\href{https://www.hes-so.ch}{University of Applied Sciences Western Switzerland}} % Your university's name and URL, this is used in the title page and abstract, print it elsewhere with \univname
% \department{Engineering} % Your department's name and URL, this is used in the title page and abstract, print it elsewhere with \deptname
\department{} % Your department's name and URL, this is used in the title page and abstract, print it elsewhere with \deptname
\faculty{Major in Information and Communication Technology} % Your research group's name and URL, this is used in the title page, print it elsewhere with \groupname
\group{Complex Information Systems} % Your faculty's name and URL, this is used in the title page and abstract, print it elsewhere with \facname

\AtBeginDocument{
\hypersetup{pdftitle=\ttitle} % Set the PDF's title to your title
\hypersetup{pdfauthor=\authorname} % Set the PDF's author to your name
\hypersetup{pdfkeywords=\keywordnames} % Set the PDF's keywords to your keywords
\hypersetup{hidelinks}
}


\begin{document}

\frontmatter % Use roman page numbering style (i, ii, iii, iv...) for the pre-content pages


% ----------------------------------------------------------------------------------------
% 	EXTERNAL CUSTOM TITLE PAGE
% ----------------------------------------------------------------------------------------

\includepdf[pages={1,2,3,4}]{title.pdf}

%----------------------------------------------------------------------------------------
%	ABSTRACT PAGE
%----------------------------------------------------------------------------------------

\begin{abstract}
\addchaptertocentry{\abstractname} % Add the abstract to the table of contents

Conversational agents (chatbots) have increased in popularity for the past two years. Businesses have a new way of communicating with their customers but must face challenges.
This thesis aims at presenting different Artifical Intelligence approaches, namely the rule-based approach and the generative approach, to constructing a chatbot.
The first chatbot was created in 1966 but since then, many improvements in Natural Language Processing and Machine Learning have led to the powerful chatbots known today as Alexa or Siri.

Recurrent Neural Networks (RNNs) have proved to be very efficient in number of tasks including machine translation, text summarization or conversational agent. Sequence-to-Sequence (Seq2Seq) is a model architecture divided into two distinct parts, namely the encoder and the decoder, both containing RNN cells.
The encoder creates an abstract representation of a sequence and the decoder uses this representation to output another sequence of text. Using this mechanism, Neural Machine Translation (NMT) model is able to translate texts in a very effective way. In addition, NMT is also able to construct a chatbot and converse.

The different experiments aim at understanding the state of the art of chatbot models and aim also at providing information on what is today feasible.
The experiments illustrated models that are able to construct grammatically correct sentences, but occasionally sentences that were inconsistent with the question.
However, Cornell Movie Dialogs corpus, the dataset used for the experiments, is between 1\% and 10\% the size of other datasets used to train more robust chatbots.
Besides from the dataset, some parameters create better models.
% For example, a model with two hidden layers performs better that a model with four or eight hidden layers.
Limiting the vocabulary led to models outputting “<unk>” tokens (i.e. out of vocabulary tokens) and showed that the perplexity and BLEU score are not always correlated and that having two methods of evaluation constructing a chatbot might reduce the chances of misunderstanding results. Reversing the input sequence to bring the beginning of the encoder’s sequence closer to the beginning of the decoder’s sequence increased the models' performances. Finally, the attention mechanism did not improve the performances; however, this might have been due to the lack of data to train the attention layer correctly.

This thesis proposes multiple ways to improve the models' performances and present alternatives to the NMT model. The thesis then advises businesses on how they should behave with this new technology.

\end{abstract}

%----------------------------------------------------------------------------------------
%	ACKNOWLEDGEMENTS
%----------------------------------------------------------------------------------------

\begin{acknowledgements}
\addchaptertocentry{\acknowledgementname} % Add the acknowledgements to the table of contents

I would like to thank my supervisor, Guido Bologna, who trusted me with this thesis. His pragmatic advices allowed me to face more easily different obstacles and his deep knowledge of the field let me to understand much clearer the theoretical background.

I also thank Nabil Abdennadher who approved a fund to use Paperspace and train my models on GPUs.

Without other people reading the thesis, there would be much more mistakes in the text. For that, I greatly thank Alex and Elodie.

I think that without her support, I would not be where I am today. I would like to thank with all my heart Elodie to be present with me every day and to help in all of this process. Maybe  moving house and adopting a cat are not the best things to do during a master's thesis but we did it and we managed to keep me working and writing this thesis. Thank you for just being who you are.

Last but not least, my parents deserve a great thank for all their support. They have always trusted me and showed me they would do anything to let me start my career with everything I would need. Thank you so much.

\end{acknowledgements}

% ----------------------------------------------------------------------------------------
% 	LIST OF CONTENTS/FIGURES/TABLES PAGES
% ----------------------------------------------------------------------------------------

\tableofcontents % Prints the main table of contents

\listoffigures % Prints the list of figures

\listoftables % Prints the list of tables

%----------------------------------------------------------------------------------------
%	ABBREVIATIONS
%----------------------------------------------------------------------------------------

\begin{abbreviations}{ll} % Include a list of abbreviations (a table of two columns)

\textbf{AIML} & \textbf{A}rtificial \textbf{I}ntelligence \textbf{M}arkup \textbf{L}anguage\\
\textbf{AMT} & \textbf{A}mazon \textbf{M}echanical \textbf{T}urk\\
\textbf{CBOW} & \textbf{C}ontinuous \textbf{B}ag \textbf{O}f \textbf{W}ords\\
\textbf{CNN} & \textbf{C}onvolutional \textbf{N}eural \textbf{N}etwork\\
\textbf{DNN} & \textbf{D}eep \textbf{N}eural \textbf{N}etwork\\
\textbf{ECM} & \textbf{E}motional \textbf{C}hatting \textbf{M}achine\\
\textbf{EOS} & \textbf{E}nd \textbf{O}f \textbf{S}equence\\
\textbf{GNMT} & \textbf{G}oogle \textbf{N}eural \textbf{M}achine \textbf{T}ranslation\\
\textbf{GRU} & \textbf{G}ated \textbf{R}ecurrent \textbf{U}nit\\
\textbf{LSTM} & \textbf{L}ong \textbf{S}hort \textbf{T}erm \textbf{M}emory\\
\textbf{ML} & \textbf{M}achine \textbf{L}earning\\
\textbf{NLP} & \textbf{N}atural \textbf{L}anguage \textbf{P}rocessing\\
\textbf{NMT} & \textbf{N}eural \textbf{M}achine \textbf{T}ranslation\\
\textbf{NN} & \textbf{N}eural \textbf{N}etwork\\
\textbf{NNLM} & \textbf{N}eural \textbf{N}etwork \textbf{L}anguage \textbf{M}odel\\
\textbf{RNN} & \textbf{R}ecurrent \textbf{N}eural \textbf{N}etwork\\
\textbf{SOS} & \textbf{S}tart \textbf{O}f \textbf{S}equence\\

\end{abbreviations}


%----------------------------------------------------------------------------------------
%	THESIS CONTENT - CHAPTERS
%----------------------------------------------------------------------------------------

\mainmatter % Begin numeric (1,2,3...) page numbering

\pagestyle{thesis} % Return the page headers back to the "thesis" style

% Include the chapters of the thesis as separate files from the Chapters folder
% Uncomment the lines as you write the chapters

% !TEX root = ../main.tex
% Chapter Introduction

\chapter{Introduction} % Main chapter title

\label{Chapter1} % Change X to a consecutive number; for referencing this chapter elsewhere, use \ref{ChapterX}

Conversational agents (chatbots) created a certain interest in the media and in the industry by bringing a new way for businesses to communicate with their customers. Although chatbots have existed since the 1970s, it is only recently that engineers have been able to provide a good enough user experience and use the chatbots as an interaction channel between the customers and the company.
Understanding and reliability are the key to attracting customers using chatbots, otherwise they will just avoid this new technology.
The goal of this thesis is to analyze the different possibilities of constructing chatbots and what is pragmatically possible today in terms of business applications. This paper then describes the next important challenges that research must address.

We see new forms of User Experience requirements. Ergonomists and information visualization specialists have worked for the past 20 years to develop best practices for desktop, web and mobile applications, thereby increasing enjoyment levels among application users.
However, with chatbots the main interaction relies on direct communication which raises new requirements.
Character development and character management are new challenges and new jobs that raise with new chatbots~\citep{1704.04579}.
Scriptwriters from the movie and theatre industries can bring their knowledge to create and imagine the ``correct'' way of conversing with a customer in relation to age, culture and gender. Thus, helping developers imagine scenarios and create the wanted emotions when interacting with the chatbot.

Chatbots are already present every day in our lives. Apple, Amazon and Google all have their advanced chatbots with different capabilities. For example, an iOS user can ask Siri to add a reminder, send a message or even, when coupled with domotics, turn off lights. Other businesses create chatbots to improve customer experiences when selling services or products, or booking a hotel or a flight.
Facebook Messenger has created a developer-friendly interface to help businesses and people create chatbots and enable them for all of the 1.37 billion daily active users Facebook has~\citep{facebook3rdquarter2017}. Dialogflow is an application that manages the backend of a chatbot and it is another tool to help businesses create chatbots rapidly.
A developer uses their services to create scenarios and prepares the answers. The application will then take care of all of the natural language processing (NLP) and provide an intelligent conversational agent.

These commercial applications are possible thanks to the research done in both NLP and deep learning fields in recent years. The two main approaches to construct a chatbot are the rule-based approach and the generative approach.
The first one has existed for more than 20 years and has limitations. Developers write down all of the possible sentences that the chatbot can encounter and the related answer. The first main issue arises when the user does not use the exact same vocabulary, or writes a sentence differently from what the developer has planned. In these instances, the chatbot is no longer able to answer.
One of the technics used for these types of chatbots is known as the Artificial Intelligence Markup Language (AIML). As a markup language, developers write questions and the adequate answers in a file.

The generative approach has appeared recently and taken advantage of the progress in Machine Learning (ML) and more particularly of Deep Neural Networks (DNN). The idea of the generative models uses the underlying idea of any ML models; given a large dataset, the model is able to learn from the data and predict a value for any input.
Basically, when a user sends a text to a chatbot, the text becomes the input of the ML model, which generates an answer based on what it learned. The chatbot then responds to the user.
The main advantage, which also answers the main issue of the rule-based approach, is that the model does not need the exact intended text. This model understands synonyms and different sentence compositions, therefore increasing the probability of understading the user's intended message.
The main disadvantage of this approach is the quantity of data needed to train a DNN model. Millions of examples are needed to train a robust generative model.

This project is a Master's thesis for the HES-SO Master of Science in Engineering (MSE) and is accounted for 30 ECTS credits.
This thesis is done for a startup based at EPFL that innovates in the way managers may interact with and manage their teams.
Understanding what is possible when using chatbots is important to help the startup refine its value proposition and build a roadmap for what will be possible tomorrow.

The first chapter introduces the thesis topic and the aim of this work. The second chapter presents a literature review of what is done with conversational agents today, and dicusses theories surrounding recurrent neural networks, neural machine translation, and advanced conversational agents.
The third chapter presents which dataset is used during the experiments and how to construct a chatbot using an end-to-end architecture. The fourth chapter discusses the results of the different experiments.
Finally, the fifth chapter compares the theory and the experiments' results and aims at proposing future work.

% !TEX root = ../main.tex
% Chapter Background information and theory

\chapter{State-of-the-art of conversational agent systems} % Main chapter title
% \chapter{Background information and theory} % Main chapter title

\label{Chapter2} % Change X to a consecutive number; for referencing this chapter elsewhere, use \ref{ChapterX}

A conversational agent (chatbot) is intended to respond to a Human by taking advantage of all its knowledge, its capacity to detect sentiments or remember the context, or its ability to search information on the web. Chatbots tend to use text as input and output format but it is also possible to use speech recognition and speech generation to allow users to speak, while conversing.

\section{Different approaches to build a conversation agent}
Chatbots are created using two different approaches, namely the \textbf{Rule-based} approach and the \textbf{Generative} approach. The Rule-based approach, as its name indicates, uses rules to understand user input and pick from a list of answers a possible one. Rule-based chatbots exist since 1966 with the developement of ELIZA chatbot in \citet{Weizenbaum:1966:ECP:365153.365168}. ELIZA's goal was to measure the psychological effects on Humans when they talk to a machine. Some of the test subjects found it really hard to believe that they were talking to a computer. ELIZA followed a script and analyzed user input to find keywords and to choose the proper answer. The workload for the developper was quite intensive and it would only take one situation that the developer might not have thought of and the user would understand he was not talking to a Human.

The Generative approach is a machine learning (ML) model that takes advantage of the recent research and technology improvements enabling deep neural networks (DNN) model to be trained on large datasets. The main difference with the rule-based approach is that the chatbot learns and establishes its own rules based on the training dataset. Thus, generative models are less entitled to misunderstand the user but since they learn themselves the output sentences, they might generate sentences with punctuation errors, grammatical errors, or even generate incomprehensible sentences. Aside the generation problems, training deep neural networks is not an easy task and require high performance hardware.

These two approaches are used in two different cases, closed-domain and open-domain conversations. A closed-domain conversation means that the chatbot answers to a particular type of questions. For example, if a company develops a chatbot to let users manage their booking details, it will only be able to answer requests about bookings and nothing else. At contrary, an open-domain chatbot is able to discuss about anything. For example, Siri the personnal assistant developed by Apple or Alexa the personnal assistant developed by Amazon are able to understand very different requests. Figure \ref{fig:types_chatbot} shows the different types of chatbots and hightlights the fact that the rule-based approach is only meant for closed-domain conversations. An open domain chatbot based on prepared scenarios and answers, following rules, is impossible due to the amount of work needed to construct such a system.

\begin{figure}
    \centering
    \includegraphics[width=.66\textwidth]{types_chatbot}
    \decoRule
    \caption{Different types of chatbot}
    \label{fig:types_chatbot}
\end{figure}

\section{Rules-based models}
In this section, two technics to build a rule-based are presented, a markup language technic and a web-based development tool.

\subsection{Artificial Intelligence Markup Language}
Artificial Intelligence Markup Language (AIML) is an XML-compliant language that simplifies the creation of rule-based chatbots \citep{wallace2005aiml}. The four main tags are described below.
\begin{itemize}
    \item \textit{<aiml>}: simply the tags that begins and ends a AIML document
    \item \textit{<category>}: tag to create an user intent
    \item \textit{<pattern>}: each category has a pattern, which is the exact phrase the user enters in the chatbot
    \item \textit{<template>}: what the chatbot answers
\end{itemize}
An example of a category, from ALICEbot, is showed in Listing~\ref{lst:aiml}.
\begin{lstlisting}[language=XML, label={lst:aiml}]
<category>
    <pattern>WHAT ARE YOU</pattern>
    <template>
        I am the latest result in artificial intelligence,
        which can reproduce the capabilities of the human brain
        with greater speed and accuracy.
    </template>
</category>
\end{lstlisting}

\citet{wallace2009anatomy} created a chatbot of approximately 41'000 categories called ALICE. A log extract of a conversation with ALICE is shows in Listing~\ref{lst:alice-log}.
\begin{lstlisting}[label={lst:alice-log}]
judge: Hi alice
ALICE: Hi there!
judge: How are you ?
ALICE: Fair to partly cloudy.
judge: Cloudy ?
ALICE: We have never talked about it before.
judge: Who is the president of the united states ?
ALICE: Obama.
\end{lstlisting}

\subsection{Development tools for rule-based conversational agents}
The interest around chatbots has risen the past year and half, as shows Figure~\ref{fig:trend_chatbot}. Different products appeared since then on the market to help businesses and privates build and deploy chatbots. One example is Dialogflow\footnote{\url{https://dialogflow.com/}} but most of the tools follow the same logic. The goal of Dialogflow is to understand the intent of the user and trigger an action. The action can be a text response provided by Dialogflow or, Dialogflow can use a webhook and send the intents to a remote server that builds the response accordingly to user's intents.
For example, a possible application would be a chatbot that tells the student where is the next lecture. Dialogflow understands the intent and forwards the request to the webhook. The webhook receives the intent and connects to the school servers to get the information and send it back to the user. Figure~\ref{fig:dialogflow} shows the flow of interactions for a chatbot developed on Dialogflow.

\begin{figure}[b]
    \centering
    \includegraphics[width=\textwidth]{trend_chatbots}
    \decoRule
    \caption[Web search interest for ``chatbot'']{Interest of the keyword ``chatbot'' in web searches for the past 5 years. Source:~Google~Trends}
    \label{fig:trend_chatbot}
\end{figure}

\begin{figure}
    \centering
    \includegraphics[width=\textwidth]{dialogflow_flow}
    \decoRule
    \caption[Dialogflow based chatbot]{Flow of interactions for a chatbot developed on Dialogflow. Since the user is speaking to a chatbot, a certain latency is acceptable but should not exceed one second.}
    \label{fig:dialogflow}
\end{figure}

The strength of these tools is that they extract the intent of the user without being restricted to an exact sentence. These tools are sometimes called hybrid because they are in their core rule-based models, but use Natural Language Processing (NLP) to understand better the user. Thus, the chatbots are more reliable and therefore, more interesting for companies like Swiss, Comcast, The Wall Street Journal or Mecedes-Benz \citep{dialogflow}. These chatbots are deployed over popular messaging platforms like Facebook Messenger or WhatsApp. Thus, the privacy of the data is an important concern. For example, Facebook's spokeswoman said that Messenger do not read the messages between people and businesses and Facebook do not use the content of Messenger conversations for any type of advertising \citep{facebook-policy}.

\section{Generative approach}
Generative conversation models are based on a learning process and need a large set of training data (e.g. \citet{1506.05869} used 62M sentences). Generative models are created using deep neural networks like Recurent Neural Network (RNN) \citep{1503.02364,1506.05869}.

\subsection{Vector representation of words}

As RNNs take real vectors as inputs, the text sequence are represented as continuous vectors, known as word embeddings. A technic called word2vec \citep{1301.3781} proposes two log-linear models based on feedforward Neural Net Language Model (NNLM) to represent efficiently words in vector space. The first model architecture of word2vec is called the Continuous Bag-of-Words (CBOW) model and follow the feedforward NNLM architecture with one hidden layer. The idea behind is that we train a model to predict a word from a sentence using $N \in \mathbb{N}$) words occuring before and N words occuring after in the sentence (e.g. in ``\textit{This corpus contains millions entities}'' and with $N=2$, the input's model are \textit{This, corpus, millions entities} and the target is \textit{contains}). Figure~\ref{fig:cbow} shows the CBOW architecture. The skip-gram architecture, represented in Figure~\ref{fig:skipgram}, takes as input the current word to predict the context. Empirical results \citep{tf.word2vec} show that the Skip-gram model is more useful for large datasets as for CBOW, it fits better on small datasets.

\begin{figure}
    \centering
    \begin{subfigure}{.45\textwidth}
        \centering
        \includegraphics[width=.8\textwidth]{word2vec_cbow}
        \caption[Continuous Bag-of-Words architecture]{CBOW architecture from word2vec models. Context is used to predict the current word.}
        \label{fig:cbow}
    \end{subfigure}
    ~
    \begin{subfigure}{.45\textwidth}
        \centering
        \includegraphics[width=.8\textwidth]{word2vec_skipgram}
        \caption[Skip-gram architecture]{Skip-gram architecture from word2vec models. Current word is used to predict the context.}
        \label{fig:skipgram}
    \end{subfigure}
    \decoRule
    \caption{Word2vec architectures. Source:~\citet{1301.3781}}
    \label{fig:word2vec}
\end{figure}

An interesting finding by \citet{1301.3781} shows that simple arithmetic operations can be performed on word embeddings to highlight and find words sharing the same relation. For example, $\mathrm{queen} - \mathrm{king} + \mathrm{man} = \mathrm{woman}$.

\subsection{Recurrent Neural Networks}
RNN is a neural network created to analyze time-series. Given a sequence $\bm{x} = (x_1, ..., x_T)$, a RNN outputs the sequence $\bm{y} = (y_1, ..., y_T)$, iterating from time $t = 1$ to $T$. Each RNN neuron has two weight matrices, one for the input and one for the output, as shown in Equation~\ref{eq:rnn_y} \citep{geron2017hands}. Let $\phi$ be an activation function and $b$ the bias vector.

\begin{align}
    \label{eq:rnn_y}
    \bm{y}_t &= \phi (\mathbf{W}_x^T \cdot \bm{x}_t + \mathbf{W}_y^T \cdot \bm{y}_{(t-1)} + b)
\end{align}

Vanilla RNN present two main problems, described in~\citet{bengio1994learning}, namely the exploding and the vanishing gradient problem. The exploding gradient problem means that the gradient grow exponentially to a point where the weights of the network, during the backpropagation are overly updated and result to a network losing all its learning. \citet{pascanu2013difficulty} proposed a simple trick to prevent exploding gradients by clipping the gradient if it becomes too big. In other words, if the gradient surpasses a certain threshold, the gradient is changed to the maximum allowed value.

The vanishing gradient problem means that the backpropagation of the gradient over the weights of the neural network has no effect on the weights themselves because the gradient is too small, meaning that the network stops learning.
Long short-term memory (LSTM) \citep{hochreiter1997lstm} is a RNN model having the capability of memorizing long distance information and being less affected by the vanishing gradient problem described in \citet{bengio1994learning}. Instead of having a single activation function, for example $\tanh$, the LSTM cell has different gates interconnected with different functions.
The following LSTM architecture is presented in \citet{1409.2329}, based on the work presented in \citet{1303.5778}. Let $h^l_t \in \mathbb{R}^n$ be a hidden state in layer $l$ in timestep $t$ and let $T_{n,m} : \mathbb{R}^n \rightarrow \mathbb{R}^m$ be an affine transformation ($Wx + b$). Figure~\ref{fig:lstm-cell} shows a graphical illustration of a LSTM cell.

\begin{align}
    \mathrm{LSTM} &: h_t^{l-1}, h_{t-1}^l, c_{t-1}^l \rightarrow h_t^l, c_t^l\\
    \begin{pmatrix}
        i\\
        f\\
        o\\
        g\\
    \end{pmatrix} &=
    \begin{pmatrix}
        \mathrm{sigm}\\
        \mathrm{sigm}\\
        \mathrm{sigm}\\
        \mathrm{sigm}\\
    \end{pmatrix}
    T_{2n,4n}
    \begin{pmatrix}
        h_t^{l-1}\\
        h_{t-1}^l\\
    \end{pmatrix}\\
    c_t^l &= f \odot c_{t-1}^l + i \odot g\\
    h_t^l &= o \odot \tanh (c_t^l)
\end{align}

\begin{figure}
    \centering
    \includegraphics[width=\textwidth]{lstm_cell}
    \decoRule
    \caption[LSTM cell]{LSTM cell. Let $h^l_t \in \mathbb{R}^n$ be a hidden state in layer $l$ in timestep $t$. Source:~\citet{1409.2329}}
    \label{fig:lstm-cell}
\end{figure}

Dropout \citep{srivastava2014dropout} is a technic often used in neural network models to reduce the chance to overfit. The dropout randomly corrupts certain weights to force the model to learn more robustly. \citet{1409.2329} successfully applied dropout in their LSTM model by applying the dropout operator on the hidden states from the previous layer.

\subsection{Neural Machine Translation}
The Neurel Machine Translation (NMT) model, proposed by~\citet{nmt-phd}, is inspired from the seq2seq architecture. Sequence-to-sequence (Seq2Seq) was first introduced by Google \citep{1409.3215} to be an end-to-end approach that makes only minimal assumptions on the sequence structure.
Researchers observed that building end-to-end deep learning systems based on discriminative neural networks often work better with data minimally preprocessed~\citep{chen2015handbook}.
In the use case of chatbots, the input and output sequences, or conversations, can have basically any length, only be tokenized, and the NN is able to learn. Seq2seq model takes text as a timeserie and use RNN as the basic cell. Seq2Seq approach decomposes the architecture in two main parts, namely the encoder and the decoder. The encoder, intialized with random weights, takes as input a sequence of text and forms a fixed-size vector representation of it.
The decoder, initialized with the last hidden state of the encoder, uses this abstract representation produced by the encoder to output the adequate output sentence. The fixed-size vector representation is meant at capturing relevant information from the input sentence.
A notable difference between seq2seq and NMT is the input of the decoder. Figure~\ref{fig:seq2seqmodel} that shows the seq2seq model architecture. In the NMT model, the decoder is fed with target output embeddings.
Figure~\ref{fig:nmt} presents the workflow of a NMT model. The example shown is a translation problem, but NMT model architecture is used also for conversational agents or text summarization \citep{tensorflow.nmt}.

\begin{figure}
    \centering
    \includegraphics[width=\textwidth]{seq2seq_model}
    \decoRule
    \caption[Seq2Seq model architecture]{Seq2Seq model architecture~\citep{1409.3215}. In the decoder part of the model, the output $h_{t-1}$ is the input $x_{t}$}
    \label{fig:seq2seqmodel}
\end{figure}

\begin{figure}
    \centering
    \includegraphics[width=.8\textwidth]{nmt_training_scheme}
    \decoRule
    \caption[Neural Machine Translation model scheme]{Neural Machine Translation model scheme \citep{tensorflow.nmt}. \textbf{(a)} \textit{Source input} refers to the input sequence tokenized to let the encoder read it token by token. \textbf{(b)} \textit{Target input} refers to the decoder input that let the decoder have the correct $t-1$ token. \textbf{(c)} \textit{Embedding layer} refers to the feedforward neural network language model that creates words' vector representation. \textbf{(d)} \textit{Encoder} refers to stacked LSTM used to build an abstract representation of the input sequence. \textbf{(e)} \textit{Decoder} refers to stacked LSTM that produce the output sequence based on the abstract representation built by the encoder and on the ``previous'' target word. \textbf{(f)} \textit{Hidden layers} refers to the number of stacked RNN layers. \textbf{(g)} \textit{Projection layer} refers to the probability distribution of the predicted word, used to calculate the loss. \textbf{(h)} \textit{Decoder output} refers to the output sequence tokenized.}
    \label{fig:nmt}
\end{figure}

Both the encoder and the decoder are RNN: LSTM \citep{1409.3215,1508.04025} or GRU \citep{1706.05125,1503.02364}. These RNN models need as input fixed-size sentences but conversations have varying length. To face this issue, the sentences are padded with $0$ vectors to reach the size of the longest sentence of the training set. However, a lot of computation and memory is required to save vectors filled with zeros and train a model. This computation leak is resolved by bucketing the training set based on the length of the sentence.
Bucketing means that the training set is partitioned into buckets containing sentences of approximately the same length. With buckets, sentences are still zero padded to the longest sentence of the bucket but it ensures that the padding is minimized.

% \textbf{Calculate the number of parameters}

% \textbf{Evaluation: loss, perplexity, cross-entropy, BLEU, being an active research topic today}
The top-layer of the decoder is fed to a softmax function to produce the predictive distribution. This distribution is used to extract the probability of every words being the target word or not, then the most probable word is chosen and output. To know how good a model works, the cross-entropy is calculated between the target and decoded distribution and used as the loss function to backpropagate the error through the network.

Perplexity~\citet{jurafsky2014speech}.

Moreover, to measure how good model actually generates words, the BLEU score is used~\citep{bleu-score}. The BLEU score measures the closeness of words between the decoded and target sentences using a n-gram precision.
% \begin{equation}
%     \mathrm{BLEU} = XX, \mathrm{BLEU} \in [0, 100]
%     \label{equ:bleu}
% \end{equation}
In NMT, the BLEU score works well because a translation is often the same even with a different context. For example, ``maison'' in French will almost always be translated as ``house'' in English.
However, when dealing with conversational agents, the context can vary a lot, from the way the user write to, for example, the weather. From this observation, \citet{1603.08023} showed that there are not currently a reliable method to measure the effectiveness of a chatbot, except the use of Human evaluators.

Once the model is trained, there are mainly two approaches to decode a sentence, known as inference, namely the greedy search or the beam search. The greedy search takes the most probable output word at each timestep and fed it in the decoder's input at $t+1$. This approach is computationnaly efficient but does not yield great quality. Figure~\ref{fig:nmt-inference} shows the inference process using a greedy search.
\begin{figure}
    \centering
    \includegraphics[width=.8\textwidth]{nmt_inference_scheme}
    \decoRule
    \caption[Neural Machine Translation inference scheme]{Neural Machine Translation inference \citep{tensorflow.nmt}. \textbf{(a)} \textit{Source input} refers to the input sequence tokenized to let the encoder read it token by token. \textbf{(b)} \textit{Embedding layer} refers to the feedforward neural network language model that creates words' vector representation. \textbf{(c)} \textit{Encoder} refers to stacked RNN model used to build an abstract representation of the input sequence. \textbf{(d)} \textit{Decoder} refers to stacked LSTM that produce the output sequence based on the abstract representation built by the encoder and on the $t-1$ output word. \textbf{(e)} \textit{Hidden layers} refers to the number of stacked LSTM layers. \textbf{(f)} \textit{Greedy search} takes the most probable output word at each timestep  \textbf{(g)} \textit{Decoder output} refers to the output sequence tokenized.}
    \label{fig:nmt-inference}
\end{figure}

The beam search is computationnaly more expensive than the greedy search but yields much better quality. The process of the beam search is to keep track of the most probable words at each time step and to feed the decoder's input with the N most probable words until the number of possibilities excess a certain threshold. At each time, the probabily of the whole sequence is calculated and only the words being in the most probable sentence are kept for $t+1$.


\subsection{Attention mechanism}
Intuitively, when one translates a sentence, one does not only read the sentence once and straightly translates it. The translation is based on the sentence overall and the translator goes back and forth in the source sentence to give the proper translation. The same intuition applies in conversations.
In basic seq2seq architecture, the model only bases its input data knowledge on the encoder's last hidden state. \citet{1508.04025} proposed a simple method to allow the model to pay attention to what matters at every timestep. They based their work on the research conducted by \citet{1409.0473} that proposed an improvement in seq2seq model performance by allowing ``\textit{the model to automatically (soft)-search for parts of a source sentence that are relevant to predicting a target word}''.

\begin{figure}
    \centering
    \includegraphics[width=.6\textwidth]{attention_basic_scheme}
    \decoRule
    \caption[Attention basic scheme]{Attention basic scheme. The decoder uses what is relevant from the input source to decode the sentence and not only the last hidden state of the encoder. Source:~\citet{youtube-nmt-attention}}
    \label{fig:attention_basic_scheme}
\end{figure}

The attention mechanism can be seen as a memory which the model retrieves information when needed. Figure~\ref{fig:attention_basic_scheme} shows the decoder taking into account attention in the decoding process. \citet{1508.04025} proposed a simple and effective approach of the attention mechanism separated into two models, namely the \textit{global attention} and \textit{local attention}, following nevertheless a common ground. At each time step $t$, the hidden-state $\bm{h}_t$ at the top layer of the stacked LSTMs is taken to construct a context vector $\bm{c}_t$.
The context vector $\bm{c}_t$ captures relevant source-side information that helps predicting the current target word $\bm{y}_t$. As for the basic NMT model, the decoder output is sent to a softmax function to output the predictive distribution. In an attention-based NMT, the current hidden state $\bm{h}_t$ is concatenated with the context vector $\bm{c}_t$ to produce an attentional hidden state, as showed in Equation~\ref{equ:atn-hidden-state}, whose fed through the softmax layer, as showed in Equation~\ref{equ:atn-hidden-state-to-softmax}.
\begin{equation}
    \bm{\tilde{h}}_t = \tanh ( \mathbf{W}_c [\bm{c}_t;\bm{h}_t])
    \label{equ:atn-hidden-state}
\end{equation}
\begin{equation}
    p( y_t | y _{<t}, x) = \mathrm{softmax(\mathbf{W_s}\bm{\tilde{h}_t})}
    \label{equ:atn-hidden-state-to-softmax}
\end{equation}
 The global and local attention models differ on how the context vector $\bm{c}_t$ is derived.

 The global attentional model takes into account all the hidden states of the decoder to produce the context vector $\bm{c}_t$. This model compares the current target hidden $\bm{h}_t$ with each source hidden state $\bm{\bar{h}}_s$ to derive a variable-length alignment vector $\bm{a}_t$ whose size equals the number of time steps on the source side.
 \begin{align}
    \bm{a}_t(s) &= \mathrm{align}(\bm{h}_t, \bm{\bar{h}}_s)\\
    \label{equ:atn-a_t}
    &= \frac{\mathrm{exp}(\mathrm{score}(\bm{h}_t, \bm{\bar{h}}_s))}{\sum_{s'} \mathrm{exp}(\mathrm{score}(\bm{h}_t, \bm{\bar{h}}_{s'}))}
 \end{align}
The score function $\mathrm{score}(\bm{h}_t, \bm{\bar{h}}_s)$ is presented in three different alternatives in~\citet{1508.04025}, as described in Equation~\ref{equ:atn-score}.
\begin{subequations}
    \label{equ:atn-score}
    \begin{align}[left ={\mathrm{score}(\bm{h}_t, \bm{\bar{h}}_s) = \empheqlbrace}]
        & \bm{h}_t^T \bm{\bar{h}}_s & \mathit{dot} \label{equ:atn-score-dot}\\
        & \bm{h}_t^T \mathbf{W_a} \bm{\bar{h}}_s & \mathit{general} \label{equ:atn-score-general}\\
        & \bm{v}_a^T \tanh ( \mathbf{W_a} [\bm{h_t}^T;\bm{\bar{h}}_s]) & \mathit{concat} \label{equ:atn-score-concat}
    \end{align}
\end{subequations}
The first alternative of the score function is described in Equation~\ref{equ:atn-score-dot} and simply performs a dot product between the decoder hidden state and encoder hidden state and tries to find the words that are similar.
The score function described in Equation~\ref{equ:atn-score-general} resembles to the ``\textit{dot}'' score function but instead, places a weight matrix in-between to let the model capture what part of the hidden states are relevant. \citet{youtube-nmt-attention} highlights the \textit{``general''} approach as working better than the two others and being the form well-adpoted by the community.
The last alternative of the score function, described in Equation~\ref{equ:atn-score-concat}, is the one proposed by \citet{1409.0473}. In this situation, the score function is like a neural network layer and the decoder and encoder hidden states are concatenated together, then multiplied by a weight matrix, fed to a sigmoid function and finally multiplied by a vector. The main concern with the last alternative is that the two hidden states are not interacting with each other.

\citet{1508.04025} approach is similar to \citet{1409.0473} but they have three main differences that simplifies and generalizes the concept. First, in \citet{1409.0473}, instead of using only the top hidden states, the model concatenates all source hidden states of the bidirectional encoder to the target hidden states.
Secondly, \citet{1409.0473} computation path requires hidden states from previous time whereas in \citet{1508.04025}, only the current time hidden states are used. Finally, as mentionned before, \citet{1409.0473} used the third function score, described in Equation~\ref{equ:atn-score-concat}.
Figure~\ref{fig:atn_global} shows the global attention model.

\begin{figure}
    \centering
    \includegraphics[width=.6\textwidth]{atn_global}
    \decoRule
    \caption[Global attention model]{Global attention model. At each time step $t$, the model derives an alignment vector $\rm{a}_t$ by comparing the source hidden states $\rm{\bar{h}}_s$ and the current target state $\rm{h}_t$ with a score function. Source:~\citet{1508.04025}}
    \label{fig:atn_global}
\end{figure}

The local attentional model, proposed by \citet{1508.04025}, uses only a certain window around a position to compute the context vector. In other terms, it does not focus on everything at each timestep. In comparison to the global attentional model, the local attentional model is less expensive since it choses the subset of the source positions per target word. At each time step t, the context vector is created as a weighted average over the set of source hidden states within the window $[p_t - D, p_t + D]$, $D$ being empirically chosen.
This windows allows the attentional vector $\rm{a}_t$ to have a fixed size ($\rm{a}_t \in \mathbb{R}^{2D+1})$. The $p_t$ parameter can be set by two different alignment methods. The \textit{monotonic alignment} simply sets $p_t = t$, assuming the source and the target sequences are somewhat aligned. The \textit{predictive alignment} learns and predicts $p_t$ in a form of a single layer feedforward neural network, as shows in Equation~\ref{equ:atn-pt-local-align}, $S$ being the sentence length.
\begin{equation}
    p_t = S \cdot \mathrm{sigmoid} ( \rm{v}_p^T \tanh ( \mathbf{W}_p \rm{h}_t))
    \label{equ:atn-pt-local-align}
\end{equation}
The position $p_t$ is taking into account to construct the alignment vector $\rm{a}_t(s)$ as defined in Equation~\ref{equ:atn-at-pt}, the standard deviation being empirically set as $\sigma = \frac{D}{2}$.
\begin{equation}
    \rm{a}_t(s) = \mathrm{align}(\rm{h}_t, \rm{\bar{h}}_s) \mathrm{exp}(- \frac{(s - p_t)^2}{2\sigma^2})
    \label{equ:atn-at-pt}
\end{equation}
Figure~\ref{fig:atn-local} shows the local attention model.
\begin{figure}
    \centering
    \includegraphics[width=.6\textwidth]{atn_local}
    \decoRule
    \caption[Local attention model]{Local attention model. The alignment vector $\rm{a}_t$ is inferred from a selected window of hidden states $\rm{\bar{h}}_t$. The position $p_t$ is learned and predicted to have a better alignment between the source hidden states $\rm{\bar{h}}_s$ and the curret target state $\rm{h}_t$. Source:~\citet{1508.04025}}
    \label{fig:atn-local}
\end{figure}

\section{Advanced conversational agents}
Despite the fact that NMT can be used as an architecture for chatbots, more advanced models use other components that empower the chatbot with more capabilities like sentiment analysis or information retrieval.

\subsection{Emotional conversational agents}
Chatbots are interesting to discuss with and sometimes they are useful, but in certain domain, like psychology, being able to react sentimentaly according to user's feelings can improve user experience and accomplish the awaited result. \citet{ecm-1704.01074} proposed an ``\textit{end-to-end framework to incorporate emotion influence in large-scale conversation generation}'' called the Emotional Chatting Machine (ECM). The model baseline is the same as the NMT model augmented with the attention mechanism, but ECM incorporates additionnaly two memory layers, an internal and an external memory, and an emotion embedding.

ECM integrates six different emotions categories, namely \textit{Anger, Disgust, Happiness, Like, Sadness, Other}. Instead of creating a chatbot capable of chosing the appropriate emotion category to generate a response, \citet{ecm-1704.01074} simplified and focused the problem by generating a different response for every emotion category. Table~\ref{tab:ecm} shows an example of different answers for a particular post.

\begin{table}
    \caption[Emotional chatting machine sample responses]{Emotional chatting machine sample responses. Source:~\citet{ecm-1704.01074}}
    \label{tab:ecm}
    \centering
    \begin{tabularx}{\textwidth}{X | l | p{5cm}}
        \toprule
        \tabhead{Post} & \tabhead{Emotion Category} & \tabhead{Response}\\
        \midrule
        % \cmidrule(lr){4-1}
        \multirow{5}{4cm}{Knowing the truth, I laughed myself to tears} & Like & Too lovely it's too lovely. \\
        & Happiness & I can't help laugh myself to tears. \\
        & Sadness & After knowing the truth, I burst into tears. \\
        & Disgust & Is this true? \\
        & Anger & What do you mean? I don't understand.\\
        \bottomrule
    \end{tabularx}
\end{table}

\subsection{MILABOT for Amazon Alexa Prize Competition}
Open domain chatbots trained using machine learning algorithms without hand-crafted rules is an ongoing research topic. On this basis, Amazon.com launched in 2016 the Amazon Alexa Prize Competition with the goal of building a socialbot, in other terms, a chatbot being able to discuss about nearly anything (e.g. sports, entertainment, politics). \citet{alexa-1709.02349} presented the best chatbot, called MILABOT, capable of having a conversation to about 14 to 16 turns (i.e. a turn is an answer from the user or the chatbot).
MILABOT is an ensemble-based machine learning system. It has several components that are trained to answer particular questions scored with an internal confidence. All the confidence score of all the modules are evaluated by the dialogue-manager to output the most appropriate answer. Figure~\ref{fig:milabot-flow} shows the dialogue manager control flow.

\begin{figure}
    \centering
    \includegraphics[width=.8\textwidth]{milabot_flow}
    \decoRule
    \caption[MILABOT dialogue manager control flow]{MILABOT dialogue manager control flow. The different modules generate their response with a confidence score. Then, the dialogue manager evaluate the different confidence scores and output the most appropriate response. Source:~\citet{alexa-1709.02349}}
    \label{fig:milabot-flow}
\end{figure}

% !TEX root = ../main.tex
% Chapter Experimental setup

\chapter{Experimental setup} % Main chapter title

\label{Chapter3} % Change X to a consecutive number; for referencing this chapter elsewhere, use \ref{ChapterX}

This chapter reviews the different experiments undertaken during this thesis. The goal of these experiments is to measure the effectiveness of more complex models against simple ones. All models are trained with the Cornell Movie Dialogs corpus.

Technologies
Keras / TensorFlow / word2vec pretrained

Data
Cornell-movie-dialogs / fr-en translate DB / Enron

\section{Different architecture for the same purpose}
The general purpose of the chatbot is to learn how to converse with other humans and to replicate the personnality of a given character from the movie dialogs. Thus, the training is done in two phases, namely the training, based on all the conversations, and the fine-tuning, based only on the conversations where the chosen character is answering. Other features like the gender of the character speaking first might lead to better results.

The model's architecture follows the Sequence-to-Sequence model for a translation problem. Instead of translating from a language to another, the model ``translates'' the question into an answer, since the input and output have the same form in both cases.

\section{Tensorflow and the NMT tutorial}
Explain what is tensorflow, why is it worth usable (calculating and derivating all the backpropagation, etc.), the idea underlying TF (graphs, tensors, etc.), what is Keras, and how it was bad to use it for chatbots (non-sparse matrix for the target output words and cross-entropy loss)
How does the NMT tutorial helps this thesis, what is it used for, what are the different parameters

\textbf{Does the LSTM cell in TensorFlow has the forget gate or not ?}

1503.02364: two weeks of training on one Tesla K40 (4.29 TFlops, M4000 2.57 TFlops, Titan V 110 TFlops) for a huge model (1k hidden states for both encoder/decoder, 620 vector space word2vec, sgd, mini-batch, vocab 40k)

\section{Hardware doing the calculations}
First I tried on Baobab, but OS too old and the server crashed while the admin was on vacations. Then, thanks to a grant of 300 CHF offered by hepia, I was able to use a dedicated Nvidia Quadro M4000 on paperspace and do all the training with it.

% !TEX root = ../main.tex
% Chapter Experiments results

\chapter{Results} % Main chapter title
\label{Chapter4} % Change X to a consecutive number; for referencing this chapter elsewhere, use \ref{ChapterX}

The main experiments' results are presented in this chapter. The different models trained and their results are explicitly detailed in Appendix~\ref{AppendixB}.


\section{Data preprocessing}
Raw string sentence is preprocessed in order to give the encoder, and the decoder, a tokenized sequence. There are different steps to prepare data for training and these steps are illustrated in Figure~\ref{fig:preprocess}. First, the raw sentence is tokenized to separate words and punctuation, the casing is not changed because it carries different information (e.g. start of sentence).

\begin{figure}
    \centering
    \includegraphics[width=\textwidth]{preprocess}
    \decoRule
    \caption[Preprocessing steps in NMT]{Preprocessing steps in an NMT model. Shapes with discontinued line refer to optional steps.}
    \label{fig:preprocess}
\end{figure}

Secondly, from the tokenized sentences, the preprocessor extracts the vocabulary (i.e. all the tokens with the number of times it appears). Table~\ref{tab:src-vocab} summarizes source vocabulary and Table~\ref{tab:tgt-vocab} summarizes target vocabulary.
Published work shows that people tend to limit the vocabulary. For example, \citet{1508.04025}, with a dataset of 4.5M sentence pairs (20 times the size of Cornell Movies Dialogs corpus), the vocabulary was limited to the 50K most frequent words. Additionally, in \citet{1506.06714}, with a dataset of 12M sentences, the vocabulary was also limited to the 50K most frequent words.
Table~\ref{tab:reduce-vocab} illustrates how both vocabularies can be reduced by using a threshold on word counts and only keep the most frequent ones.
In~\citet{1506.05869}, two different datasets were used. The first dataset contained 33M sentences and the vocabulary was limited to 20K words, and in the second dataset, composed of 88M sentences, the vocabulary was limited to 100K words.

\begin{table}
    \centering
    \caption[Source vocabulary analysis]{Source vocabulary analysis. The number of unique words:~\num{64839}. For example, it only takes \num{12603} words (\num{19.43}\% of the vocabulary) to reach \num{97}\% of the source sentences word count.}
    \label{tab:src-vocab}
    \begin{tabular}{rrr|r}
        \toprule
        \tabhead{Total count \%} & \tabhead{Unique words} & \tabhead{Vocabulary \%} & \tabhead{Current count value}\\
        \midrule
        \num{90.0} & \num{1974} & \num{3.04} & \num{82}\\
        \num{97.0} & \num{12603} & \num{19.43} & \num{7}\\
        \num{98.0} & \num{19259} & \num{29.70} & \num{4}\\
        \num{99.0} & \num{32910} & \num{50.76} & \num{2}\\
        \num{99.9} & \num{61584} & \num{94.98} & \num{1}\\
        \bottomrule
    \end{tabular}
\end{table}

\begin{table}
    \centering
    \caption[Target vocabulary analysis]{Target vocabulary analysis. Number of unique words:~\num{65875}}
    \label{tab:tgt-vocab}
    \begin{tabular}{rrr|r}
        \toprule
        \tabhead{Total count \%} & \tabhead{Unique words} & \tabhead{Vocabulary \%} & \tabhead{Current count value}\\
        \midrule
        \num{90.0} & \num{1942} & \num{2.94} & \num{86}\\
        \num{97.0} & \num{12577} & \num{19.09} & \num{7}\\
        \num{98.0} & \num{19321} & \num{29.33} & \num{4}\\
        \num{99.0} & \num{33215} & \num{50.42} & \num{2}\\
        \num{99.9} & \num{62520} & \num{94.90} & \num{1}\\
        \bottomrule
    \end{tabular}
\end{table}

\begin{table}
    \centering
    \caption[Vocabulary reduction]{Limiting the vocabulary size based on the word counts.}
    \label{tab:reduce-vocab}
    \begin{tabular}{rrrr}
        \toprule
        \tabhead{Min count} & \tabhead{Unique words} & \tabhead{Vocabulary \%} & \tabhead{Total count \%}\\
        \midrule
        \multicolumn{4}{l}{\textit{Source vocabulary}}\\
        \num{3} & \num{24347} & \num{37.55} & \num{98.47}\\
        \num{5} & \num{16456} & \num{25.38} & \num{97.66}\\
        \num{10} & \num{9881} & \num{15.24} & \num{96.34}\\
        \hline
        \multicolumn{4}{l}{\textit{Target vocabulary}}\\
        \num{3} & \num{24736} & \num{37.37} & \num{98.49}\\
        \num{5} & \num{16724} & \num{25.39} & \num{97.69}\\
        \num{10} & \num{9947} & \num{15.10} & \num{96.38}\\
        \bottomrule
    \end{tabular}
\end{table}

The final step of the preprocessing is different from the source and target set. Source sentences can be reversed, word-wise, to improve performance as mentioned in \citet{1409.3215}.
The last encoder's hidden state is the beginning of the sentence and it allows the decoder to be closer to the start of the sequence.
Target sentences need to be padded with Start-Of-Sequence (SOS) and End-Of-Sequence (EOS) tokens to let the decoder know when the sequence starts and stops.

The source and target vocabularies present similar statistics because of the fact that most of the conversations present in the corpus have multiple turns. Thus, for example, conversation \code{[A, B, C, D]} is fed into the training set as \code{[(A, B), (B, C), (C, D)]} which makes \code{[B, C]} appear in both source and target sets.

\section{Tensorflow and Neural Machine Translation tutorial}
The models' training was done using the NMT tutorial script by~\citet{tensorflow.nmt}. The authors use the Tensorflow~\citep{tensorflow2015-whitepaper} open-source library made ``\textit{for numerical computation using data flow graphs}''. Tensorflow simplifies development tasks in a machine-learning project. The developer creates his model as a graph, where nodes are operations and edges are matrices (tensors). From this graph definition of the mathematical model, Tensorflow automatically calculates the gradients and the derivatives needed for backpropagation.
Another advantage to use Tensorflow is that it runs on either CPUs or GPUs without changing a line of code. The same code is used when the engineer works locally on its machine and when the model is trained on servers with dedicated GPU cards.

The NMT tutorial script was written to let people create and train an NMT model without having to spend time coding and debugging the algorithms. The script is part of a 5 years long Ph.D. thesis \citep{nmt-phd}. The script has 65 different arguments, detailed in Appendix~\ref{AppendixA}, going from the type of RNN to the attention's type of architecture. Despite the lack of coding, if one uses the script, it has to fully understand all the arguments and how they modify the model in order to train  it correctly.

\section{Experiments}

There are five main experiments that establish at the end the model parameters to create a chatbot based on the Cornell Movie Dialogs corpus. Among all the experiments, except if specified, the dataset (i.e. training, development and test sets), model parameters specified in Table~\ref{tab:runs-shared-param} and the seed for model initialization are kept unchanged to avoid biases and to measure effectively the different models' performances.

\begin{table}
    \centering
    \caption[Runs' shared parameters]{Runs' shared parameters amongst all the experiments.}
    \label{tab:runs-shared-param}
    \begin{tabular}{ll p{.5\textwidth}}
        \toprule
        \tabhead{Parameter} & \tabhead{Value} & \tabhead{Comment}\\
        \midrule
        \code{-{}-num\_train\_steps} & \num{12000} & Number of training steps\\
        \code{-{}-steps\_per\_stats} & \num{100} & - \\
        \code{-{}-random\_seed} & \num{27} & - \\
        \code{-{}-metrics} & bleu & BLEU score is used to evaluate the parameter text output\\
        \code{-{}-dropout} & \num{0.2} & - \\
        \code{-{}-optimizer} & sgd & Use Stochastic Gradient Descent \\
        \code{-{}-batch\_size} & 128 & - \\
        \code{-{}-num\_buckets} & 5 & Number of buckets to group sentences of approximatively same length \\
        \bottomrule
    \end{tabular}
\end{table}

The models' name follows a writing convention to allow the reader to instantly know the parameters used to train a particular model. All the names have the same skeleton, the parameters are separated by a dash. Table~\ref{tab:run-name-desc} illustrates the different fields by decomposing as example ``run-20171212014119-norev-l8-u256-clstm-lr01-bw0-voc5''.
\begin{table}
    \centering
    \caption[Run name description]{Run name description. Example with run-20171212014119-norev-l8-u256-clstm-lr01-bw0-voc5.}
    \label{tab:run-name-desc}
    \begin{tabular}{l p{.7\textwidth}}
        \toprule
        \tabhead{Name part} & \tabhead{Description}\\
        \midrule
        \code{20171212014119} & Dataset the model was trained on \\
        \code{[no]rev} & If the input sequence is reversed or not \\
        \code{l8} & How many layers the model has, here 8 \\
        \code{u256} & How many units the model has, here 256 \\
        \code{clstm} & RNN used, here LSTM \\
        \code{lr01} & Learning rate, here $0.1$ \\
        \code{bw0} & Beam width, here 0 so the decoding process uses greedy search \\
        \code{voc5} & The count threshold used to filter out rare words from the vocabulary. If ``voca'', all the vocabulary is used \\
        \bottomrule
    \end{tabular}
\end{table}


\subsection{Experiment 1 - Neural network parameters}
This experiment is focused on the neural network parameters. The exhaustive list of the parameters tested is illustrated in Table~\ref{tab:run01-params}. 72 models were trained, by combining the parameters, on the whole vocabulary of the training set.

\begin{table}
    \centering
    \caption[Experiment 1 parameters]{Experiment 1 parameters measured.}
    \label{tab:run01-params}
    \begin{tabular}{ll p{.5\textwidth}}
        \toprule
        \tabhead{Parameter} & \tabhead{Values} & \tabhead{Comment}\\
        \midrule
        \code{-{}-num\_layers} & 2, 4, 8 & Network depth \\
        \code{-{}-num\_units} & 64, 128, 256 & Network size and emdeddings dimensionnality\\
        \code{-{}-cell} & lstm, gru & - \\
        \code{-{}-learning\_rate} & 1.0, 0.1 & - \\
        \code{-{}-beam\_width} & 0, 3 & If 0, decoding uses greedy search. If 3, decoding uses beam search\\
        \bottomrule
    \end{tabular}
\end{table}

Table~\ref{tab:run01-describe} shows the statistics for all the 72 models. The first surprising thing to notice is the low BLEU scores. The maximum development BLEU is \num{0.4} and the maximum test BLEU is \num{0.2}, knowing that BLEU score $\in [0, 100]$.
The second thing to notice is that the perplexities vary significatively. The difference between the mininimum and the maximum is of \num{342}. However, the third quartile is close to the maximum showing that at least 25\% of the different parameters combinations lead to bad performance.
The average time of training is about 70 minutes and the total training time is 85 hours (3 days and 13 hours). The models presenting the minimum perplexities and maximum BLEU score are described in Table~\ref{tab:run01-best-models}.
\begin{table}
    \centering
    \caption[Experiment 1 performance statistics]{Experiment 1 performance statistics}
    \label{tab:run01-describe}
    \begin{tabular}{lrrrrr}
\toprule
{} &   \tabhead{dev\_bleu} &     \tabhead{dev\_ppl} &  \tabhead{test\_bleu} &    \tabhead{test\_ppl} &         \tabhead{time} \\
\midrule
mean  &   0.023611 &  270.738194 &   0.009722 &  234.481389 &  4262.569444 \\
std   &   0.072176 &  166.934694 &   0.041655 &  153.184329 &  1144.635582 \\
min   &   0.000000 &   86.710000 &   0.000000 &   67.680000 &  2135.000000 \\
25\%   &   0.000000 &  107.485000 &   0.000000 &   85.395000 &  3447.250000 \\
50\%   &   0.000000 &  172.050000 &   0.000000 &  142.400000 &  4188.000000 \\
75\%   &   0.000000 &  458.630000 &   0.000000 &  406.577500 &  5123.750000 \\
max   &   0.400000 &  462.410000 &   0.200000 &  409.810000 &  6731.000000 \\
\bottomrule
\end{tabular}

\end{table}
\begin{table}
    \centering
    \caption[Experiment 1 best metrics]{Experiment 1 best metrics. The \textbf{maximum} value of the BLEU score and the \textbf{minimum} value of the perplexity are chosen.}
    \label{tab:run01-best-models}
    \begin{tabular}{lrl}
        \toprule
        \tabhead{Metric} & \tabhead{Best} & \tabhead{Model}\\
        \midrule
        dev\_ppl & \num{86.71} & run-20171212014119-norev-l2-u256-clstm-lr10-bw3-voca\\
        dev\_bleu & \num{0.4} & run-20171212014119-norev-l2-u64-cgru-lr01-bw0-voca\\
        test\_ppl & \num{67.80} & run-20171212014119-norev-l2-u256-cgru-lr10-bw0-voca\\
        test\_bleu & \num{0.2} & run-20171212014119-norev-l2-u128-clstm-lr10-bw0-voca\\
        \bottomrule
    \end{tabular}
\end{table}

Since Table~\ref{tab:run01-best-models} shows four different models for the four metrics and no decision can be made on which parameters trains the best model, Table~\ref{tab:run01-best-models-details} shows all the metrics for all the four models. There are no apparent best parameters based on these measures. The only shared parameter is that two layers seem to work the best.

\begin{table}
    \centering
    \caption[Experiment 1 best models]{Experiment 1 best models with development and testing perplexities and BLEU score metrics.}
    \label{tab:run01-best-models-details}
    \begin{tabular}{rrrr}
        \toprule
        \tabhead{dev\_ppl} & \tabhead{dev\_bleu} & \tabhead{test\_ppl} & \tabhead{test\_bleu}\\
        \midrule
        \multicolumn{4}{l}{\textit{run-20171212014119-norev-l2-u256-clstm-lr10-bw3-voca}}\\
        \num{86.71} & \num{0.00} & \num{68.89} & \num{0.00}\\
        \hline

        \multicolumn{4}{l}{\textit{run-20171212014119-norev-l2-u64-cgru-lr01-bw0-voca}}\\
        \num{172.04} & \num{0.40} & \num{141.75} & \num{0.20}\\
        \hline

        \multicolumn{4}{l}{\textit{run-20171212014119-norev-l2-u256-cgru-lr10-bw0-voca}}\\
        \num{86.73} & \num{0.10} & \num{67.68} & \num{0.00}\\
        \hline

        \multicolumn{4}{l}{\textit{run-20171212014119-norev-l2-u128-clstm-lr10-bw0-voca}}\\
        \num{94.54} & \num{0.10} & \num{75.34} & \num{0.20}\\

        \bottomrule
    \end{tabular}
\end{table}

The question still to be answered is which parameter influences the most the performance of the model.
Figure~\ref{fig:res_run01_bw_ppl}, Figure~\ref{fig:res_run01_c_ppl}, Figure~\ref{fig:res_run01_l_ppl}, Figure~\ref{fig:res_run01_lr_ppl} and Figure~\ref{fig:res_run01_u_ppl} illustrates the models' test perplexity against the different parameters.
The models number is not a model ID. If two different runs have the same number, it means that they share all the parameters except the one analyzed.
Based on the figures, the beam-width and the RNN does not influence the performance of the model. However, the greater learning rate and the greater number of units improve the model. At contrary, a deeper network (i.e. more layers) decreases the performance.

The best model chosen for this experiment is ``run-20171212014119-norev-l2-u128-clstm-lr10-bw0-voca''.

\begin{landscape}
\begin{figure}
    \centering
    \includegraphics[width=\textheight]{res_run01_bw_ppl}
    \decoRule
    \caption[Results experiment 1 BW-PPL]{Experiment 1 results. Each color represents a beam-width value, x-axis represents the models and y-axis the test perplexity.}
    \label{fig:res_run01_bw_ppl}
\end{figure}
\begin{figure}
    \centering
    \includegraphics[width=\textheight]{res_run01_c_ppl}
    \decoRule
    \caption[Results experiment 1 C-PPL]{Experiment 1 results. Each color represents a RNN, x-axis represents the models and y-axis the test perplexity.}
    \label{fig:res_run01_c_ppl}
\end{figure}
\begin{figure}
    \centering
    \includegraphics[width=\textheight]{res_run01_l_ppl}
    \decoRule
    \caption[Results experiment 1 L-PPL]{Experiment 1 results. Each color represents a layer value, x-axis represents the models and y-axis the test perplexity.}
    \label{fig:res_run01_l_ppl}
\end{figure}
\begin{figure}
    \centering
    \includegraphics[width=\textheight]{res_run01_lr_ppl}
    \decoRule
    \caption[Results experiment 1 LR-PPL]{Experiment 1 results. Each color represents a learning rate value, x-axis represents the models and y-axis the test perplexity.}
    \label{fig:res_run01_lr_ppl}
\end{figure}
\begin{figure}
    \centering
    \includegraphics[width=\textheight]{res_run01_u_ppl}
    \decoRule
    \caption[Results experiment 1 U-PPL]{Experiment 1 results. Each color represents an unit value, x-axis represents the models and y-axis the test perplexity.}
    \label{fig:res_run01_u_ppl}
\end{figure}
\end{landscape}

\subsection{Experiment 2 - Limit the vocabulary}
Considering the time needed to train the models in experiment 1, and the fact that the NMT script~\citep{tensorflow.nmt} was thought to use LSTM as RNN model, models using GRU are not tested from experiment 2. This experiment took 31 hours (1 day and 7 hours) and focuses on the importance of the vocabulary in a chatbot model.
Only words appearing more than five times in the training set are kept. As mentioned in Table~\ref{tab:reduce-vocab}, the source vocabulary for a count threshold of 5 contains \num{16456} unique words and the target vocabulary contains \num{16724} unique words.

By comparing Table~\ref{tab:run02-describe} and Table~\ref{tab:run01-describe}, the mean test perplexity in experiment 2 reduced by 52.53\% and the minimum test perplexity reduced by 25.33\%.
\citet{nlp-jurasky-4.4} stated in their defition of perplexity that models can be measured correctly only if their vocabulary is the same. Intuitively, seeing the perplexities reduced is only the result of a smaller vocabulary since the model is less proabable of mistaken. Despite this, the BLEU score kept the same value as during experiment 1.
Limiting the vocabulary might lead to better results, but Human evaluation is needed to verify this assumption. Table~\ref{tab:run02-best-models-details} shows the best models for all the four metrics.

\begin{table}
    \centering
    \caption[Experiment 2 performance statistics]{Experiment 2 performance statistics.}
    \label{tab:run02-describe}
    \begin{tabular}{lrrrrr}
\toprule
{} &   \tabhead{dev\_bleu} &     \tabhead{dev\_ppl} &  \tabhead{test\_bleu} &    \tabhead{test\_ppl} &         \tabhead{Time [s]} \\
\midrule
mean  &   \num{0.030556} &  \num{188.909722} &   \num{0.008333} &  \num{164.751667} &  \num{3107.194444} \\
std   &   \num{0.085589} &  \num{120.715950} &   \num{0.036839} &  \num{111.924972} &  \num{1144.878105} \\
min   &   \num{0.000000} &   \num{64.410000} &   \num{0.000000} &   \num{50.620000} &  \num{1609.000000} \\
25\%   &   \num{0.000000} &   \num{79.350000} &   \num{0.000000} &   \num{63.402500} &  \num{2326.750000} \\
50\%   &   \num{0.000000} &  \num{117.650000} &   \num{0.000000} &   \num{95.975000} &  \num{2602.000000} \\
75\%   &   \num{0.000000} &  \num{334.432500} &   \num{0.000000} &  \num{299.327500} &  \num{4194.750000} \\
max   &   \num{0.400000} &  \num{336.310000} &   \num{0.200000} &  \num{301.840000} &  \num{5204.000000} \\
\bottomrule
\end{tabular}

\end{table}
\begin{table}
    \centering
    \caption[Experiment 2 best models]{Experiment 2 best models with development and testing perplexities and BLEU score metrics.}
    \label{tab:run02-best-models-details}
    \begin{tabular}{rrrr}
        \toprule
        \tabhead{dev\_ppl} & \tabhead{dev\_bleu} & \tabhead{test\_ppl} & \tabhead{test\_bleu}\\
        \midrule
        \multicolumn{4}{l}{\textit{run-20171212014119-norev-l2-u256-clstm-lr10-bw3-voc5}}\\
        \num{64.41} & \num{0.00} & \num{51.19} & \num{0.00}\\
        \hline

        \multicolumn{4}{l}{\textit{run-20171212014119-norev-l2-u256-clstm-lr10-bw0-voc5}}\\
        \num{64.66} & \num{0.10} & \num{50.62} & \num{0.00}\\
        \hline

        \multicolumn{4}{l}{\textit{run-20171212014119-norev-l4-u256-clstm-lr01-bw0-voc5}}\\
        \num{121.43} & \num{0.40} & \num{98.91} & \num{0.00}\\
        \hline

        \multicolumn{4}{l}{\textit{run-20171212014119-norev-l2-u64-clstm-lr01-bw0-voc5}}\\
        \num{115.1} & \num{0.30} & \num{94.60} & \num{0.20}\\

        \bottomrule
    \end{tabular}
\end{table}

Figure~\ref{fig:res_run02_bw_ppl}, Figure~\ref{fig:res_run02_l_ppl}, Figure~\ref{fig:res_run02_lr_ppl} and Figure~\ref{fig:res_run02_u_ppl} show that the model parameters act in the same way on models' performances than during experiment 1. The beam-width and the RNN parameters do not have an impact on the model performance. At contrary, limiting the number of layers, representing tokens with high-dimensionality vectors and preferring a higher learning rate tend to decrease the models' perplexities.

The best model chosen for this experiment is ``run-20171212014119-norev-l2-u256-clstm-lr10-bw0-voc5''. The BLEU scores are still very low, thus the best model decision is mainly based on the test perplexity.

\begin{landscape}
\begin{figure}
    \centering
    \includegraphics[width=.95\textheight]{res_run02_bw_ppl}
    \decoRule
    \caption[Results experiment 2 BW-PPL]{Experiment 2 results. Each color represents a beam-width value, x-axis represents the experiment number and y-axis the test perplexity.}
    \label{fig:res_run02_bw_ppl}
\end{figure}
% \begin{figure}
%     \centering
%     \includegraphics[width=\textheight]{res_run02_c_ppl}
%     \decoRule
%     \caption[Results experiment 2 C-PPL]{Experiment 2 results. Each color represents a RNN, x-axis represents the models and y-axis the test perplexity.}
%     \label{fig:res_run02_c_ppl}
% \end{figure}
\begin{figure}
    \centering
    \includegraphics[width=.95\textheight]{res_run02_l_ppl}
    \decoRule
    \caption[Results experiment 2 L-PPL]{Experiment 2 results. Each color represents a layer value, x-axis represents the experiment number and y-axis the test perplexity.}
    \label{fig:res_run02_l_ppl}
\end{figure}
\begin{figure}
    \centering
    \includegraphics[width=.95\textheight]{res_run02_lr_ppl}
    \decoRule
    \caption[Results experiment 2 LR-PPL]{Experiment 2 results. Each color represents a learning rate value, x-axis represents the experiment number and y-axis the test perplexity.}
    \label{fig:res_run02_lr_ppl}
\end{figure}
\begin{figure}
    \centering
    \includegraphics[width=.95\textheight]{res_run02_u_ppl}
    \decoRule
    \caption[Results experiment 2 U-PPL]{Experiment 2 results. Each color represents an unit value, x-axis represents the experiment number and y-axis the test perplexity.}
    \label{fig:res_run02_u_ppl}
\end{figure}
\end{landscape}


\subsection{Experiment 3 - Reversing input sequence}
This experiment focuses on the benefits from inverting the input sequences. This experiment keeps the vocabulary limitation from experiment 2 because multiple papers limited the vocabulary \citep{ecm-1704.01074,1503.02364,1506.05869,1703.01619,tensorflow.nmt} and the time allowed for the experiments is restricted.

By comparing Table~\ref{tab:run02-describe} from experiment 2 and Table~\ref{tab:run03-describe} describing the experiment 3 global performance statistics, there are some improvements. First, although the maximum development BLEU score is smaller by \num{0.1} points, the mean score improved of \num{0.01} points meaning that reversing input sequence might stabilize the models' performances. Secondly, the minimum test perplexity improved by \num{0.27} points and the mean test perplexity improved by \num{4.1} points.

\begin{table}
    \centering
    \caption[Experiment 3 performance statistics]{Experiment 3 performance statistics.}
    \label{tab:run03-describe}
    \begin{tabular}{lrrrrr}
\toprule
{} &   \tabhead{dev\_bleu} &     \tabhead{dev\_ppl} &  \tabhead{test\_bleu} &    \tabhead{test\_ppl} &         \tabhead{Time [s]} \\
\midrule
mean  &   \num{0.036111} &  \num{184.689167} &   \num{0.008333} &  \num{160.852778} &  \num{3476.611111} \\
std   &   \num{0.076168} &  \num{118.469654} &   \num{0.036839} &  \num{109.711937} &   \num{969.338811} \\
min   &   \num{0.000000} &   \num{64.140000} &   \num{0.000000} &   \num{50.350000} &  \num{2121.000000} \\
25\%   &   \num{0.000000} &   \num{78.042500} &   \num{0.000000} &   \num{62.440000} &  \num{2601.500000} \\
50\%   &   \num{0.000000} &  \num{117.200000} &   \num{0.000000} &   \num{96.055000} &  \num{3446.000000} \\
75\%   &   \num{0.025000} &  \num{334.207500} &   \num{0.000000} &  \num{299.252500} &  \num{4125.750000} \\
max   &   \num{0.300000} &  \num{342.640000} &   \num{0.200000} &  \num{305.070000} &  \num{5738.000000} \\
\bottomrule
\end{tabular}

\end{table}

Figure~\ref{fig:res_run03_bw_ppl}, Figure~\ref{fig:res_run03_l_ppl}, Figure~\ref{fig:res_run03_lr_ppl} and Figure~\ref{fig:res_run03_u_ppl} show that the model parameters act in the same way on models' performances than during experiment 1 and experiment 2. When performing the best, the following parameters are used.
\begin{itemize}
    \item Learning rate: 1.0
    \item Number of units: 256
    \item Number of layers: 2
\end{itemize}

Table~\ref{tab:run03-best-models-details} shows the best models for each one of the four metrics. There are only three models because the model ``run-20171212014119-rev-l2-u256-clstm-lr10-bw0-voc5'' has the best development and test perplexity. Thus, it is the best model chosen for this experiment.

\begin{table}
    \centering
    \caption[Experiment 3 best models]{Experiment 3 best models with development and testing perplexity and BLEU score metrics.}
    \label{tab:run03-best-models-details}
    \begin{tabular}{rrrr}
        \toprule
        \tabhead{dev\_ppl} & \tabhead{dev\_bleu} & \tabhead{test\_ppl} & \tabhead{test\_bleu}\\
        \midrule
        \multicolumn{4}{l}{\textit{run-20171212014119-rev-l2-u256-clstm-lr10-bw0-voc5}}\\
        \num{64.41} & \num{0.10} & \num{50.35} & \num{0.00}\\
        \hline

        \multicolumn{4}{l}{\textit{run-20171212014119-rev-l2-u64-clstm-lr10-bw0-voc5}}\\
        \num{76.50} & \num{0.10} & \num{61.02} & \num{0.00}\\
        \hline

        \multicolumn{4}{l}{\textit{run-20171212014119-rev-l2-u64-clstm-lr01-bw0-voc5}}\\
        \num{114.62} & \num{0.30} & \num{94.32} & \num{0.20}\\

        \bottomrule
    \end{tabular}
\end{table}

\begin{landscape}
\begin{figure}
    \centering
    \includegraphics[width=\textheight]{res_run03_bw_ppl}
    \decoRule
    \caption[Results experiment 3 BW-PPL]{Experiment 3 results. Each color represents a beam-width value, x-axis represents the models and y-axis the test perplexity.}
    \label{fig:res_run03_bw_ppl}
\end{figure}
% \begin{figure}
%     \centering
%     \includegraphics[width=\textheight]{res_run03_c_ppl}
%     \decoRule
%     \caption[Results experiment 3 C-PPL]{Experiment 3 results. Each color represents a RNN, x-axis represents the models and y-axis the test perplexity.}
%     \label{fig:res_run03_c_ppl}
% \end{figure}
\begin{figure}
    \centering
    \includegraphics[width=\textheight]{res_run03_l_ppl}
    \decoRule
    \caption[Results experiment 3 L-PPL]{Experiment 3 results. Each color represents a layer value, x-axis represents the models and y-axis the test perplexity.}
    \label{fig:res_run03_l_ppl}
\end{figure}
\begin{figure}
    \centering
    \includegraphics[width=\textheight]{res_run03_lr_ppl}
    \decoRule
    \caption[Results experiment 3 LR-PPL]{Experiment 3 results. Each color represents a learning rate value, x-axis represents the models and y-axis the test perplexity.}
    \label{fig:res_run03_lr_ppl}
\end{figure}
\begin{figure}
    \centering
    \includegraphics[width=\textheight]{res_run03_u_ppl}
    \decoRule
    \caption[Results experiment 3 U-PPL]{Experiment 3 results. Each color represents an unit value, x-axis represents the models and y-axis the test perplexity.}
    \label{fig:res_run03_u_ppl}
\end{figure}
\end{landscape}

\subsection{Experiment 4 - Attention mechanism}
In their results, \citet{1508.04025} show that for long sentences (between 30 and 70 words), the attention models keep track of the information whereas models without attention show catastrophic results. Moreover, even for short sentences, attention models outperform the model without attention. From that observation, it would be interesting to measure if the attention mechanism increases the performance of the chatbot as well. However, \citet{1506.05869}, building a neural conversation machine, reported that adding the attention mechanism \citep{1409.0473} did not improve the perplexity on either training or validation sets.

There are two different attention mechanisms tested in this experiment, namely the ``luong'' (described in Chapter~\ref{Chapter2}, \citet{nmt-phd}) and ``scaled-luong''. The second attention mechanism is almost the same as the first with the exception that it is using a weight normalization inspired from \citet{1602.07868}. Reparameterizing the weights allow the model to converge faster in a stochastic gradient descent.

Experiment 4 only keeps the best value for the parameters \textit{``learning rate''} and \textit{``layers''} found in the three previous experiments. The learning is fixed to \num{1.0} and the number of layers is fixed to \num{2}.
Moreover, since the number of units kept improving the results, instead of testing on ``64'', ``128'' and ``256'', experiment 4 uses the values ``256'', ``384'' and ``512''. In total, during this experiment, 12 models were trained, in 9 hours.

Table~\ref{tab:run04-describe} describes the statistics around the experiment 4 training. Since two parameters were chosen following the three previous experiments, the standard deviation for all the metrics is significantly reduced. Comparing with the results from the previous experiment, the mean test perplexity increased of \num{2.21} points and the test BLEU metric is now only \num{0.00}. The attention mechanism increases the number of learnable parameters and this might mean that there is not enough data to also train the attention weights.

\begin{table}
    \centering
    \caption[Experiment 4 performance statistics]{Experiment 4 performance statistics.}
    \label{tab:run04-describe}
    \begin{tabular}{lrrrrr}
\toprule
{} &   \tabhead{dev\_bleu} &     \tabhead{dev\_ppl} &  \tabhead{test\_bleu} &    \tabhead{test\_ppl} &         \tabhead{Time [s]} \\
\midrule
mean  &   \num{0.025000} &  \num{67.340000} &        \num{0.0} &  \num{53.509167} &  \num{2796.666667} \\
std   &   \num{0.045227} &   \num{0.840757} &        \num{0.0} &   \num{0.914444} &   \num{267.487411} \\
min   &   \num{0.000000} &  \num{66.070000} &        \num{0.0} &  \num{52.560000} &  \num{2448.000000} \\
25\%   &   \num{0.000000} &  \num{66.835000} &        \num{0.0} &  \num{52.762500} &  \num{2637.750000} \\
50\%   &   \num{0.000000} &  \num{67.220000} &        \num{0.0} &  \num{53.220000} &  \num{2790.000000} \\
75\%   &   \num{0.025000} &  \num{68.050000} &        \num{0.0} &  \num{54.210000} &  \num{2831.750000} \\
max   &   \num{0.100000} &  \num{68.620000} &        \num{0.0} &  \num{55.270000} &  \num{3389.000000} \\
\bottomrule
\end{tabular}

\end{table}

Figure~\ref{fig:res_run04_bw_ppl}, Figure~\ref{fig:res_run04_u_ppl} and Figure~\ref{fig:res_run04_at_ppl} illustrates the models behaviour following different parameters' values. As for all other previous experiments, the beam width does not decrease or increase by itself the performance of the models. As suspected, adding more units to the model lead to better results. However, the difference in performance from experiments 1, 2 or 3 between 64 units and 256 units is now less visible between 256 units and 512 units. Finally, the different attention strategies do not seem to have an impact themselves on the models' performances.

The model having the best test perplexity is ``run-20171212014119-rev-l2-u512-clstm-lr10-bw0-atluong''.

\begin{landscape}
\begin{figure}
    \centering
    \includegraphics[width=.95\textheight]{res_run04_bw_ppl}
    \decoRule
    \caption[Results experiment 4 BW-PPL]{Experiment 4 results. Each color represents a beam-width value, x-axis represents the models and y-axis the test perplexity.}
    \label{fig:res_run04_bw_ppl}
\end{figure}
\begin{figure}
    \centering
    \includegraphics[width=.95\textheight]{res_run04_u_ppl}
    \decoRule
    \caption[Results experiment 4 U-PPL]{Experiment 4 results. Each color represents an unit value, x-axis represents the models and y-axis the test perplexity.}
    \label{fig:res_run04_u_ppl}
\end{figure}
\begin{figure}
    \centering
    \includegraphics[width=.95\textheight]{res_run04_at_ppl}
    \decoRule
    \caption[Results experiment 4 AT-PPL]{Experiment 4 results. Each color represents a different attention mechanism implementation, x-axis represents the models and y-axis the test perplexity. ``\textit{atl}'' means the attention implementation used is ``luong'' and ``\textit{ats}'' means the attention implementation used is ``scaled\_luong''.}
    \label{fig:res_run04_at_ppl}
\end{figure}
\end{landscape}

\subsection{Experiment 5 - Other dataset splitting ratio}
Since the BLEU score was significantly small for the four previous experiments, this experiment aims at training models using dataset with the 80-10-10 ratio for the training, development and testing sets. The training set contains \num{177023} conversations, the development set contains \num{22095} conversations and the test set contains \num{22161} conversations.

The parameters used for this experiment are the same as in experiment 4, except the number of units fixed to ``512''. In total, this experiment trained 4 different models, in 17 hours.

Table~\ref{tab:run05-describe} shows the different performances obtained during the experiment. By comparing it with Table~\ref{tab:run04-describe}, from experiment 4, the mean test perplexity increased by \num{13.35} points and the min test perplexity increased by \num{12.29} points. However, the BLEU score for both development and test are better than during the previous experiment. Both mean BLEU score is at \num{0.15} point, which is \num{0.12} higher than experiment 4 development BLEU score.

Although the perplexities are worse than in experiment 4, BLEU scores are better and it might mean that besides the lack of evidence that BLEU score is useful to evaluate chatbots, the other dataset might have too small development and test sets.

Besides the evaluation metrics, the time needed to train a model is longer. The mean time is \num{15433.25} whereas, in experiment 4, the mean time was \num{2796.67}, being around 5 times faster than experiment 5.

\begin{table}
    \centering
    \caption[Experiment 5 performance statistics]{Experiment 5 performance statistics.}
    \label{tab:run05-describe}
    \begin{tabular}{lrrrrr}
\toprule
{} &  \tabhead{dev\_bleu} &   \tabhead{dev\_ppl} &  \tabhead{test\_bleu} &   \tabhead{test\_ppl} &          \tabhead{Time [s]} \\
\midrule
mean  &  \num{0.150000} &  \num{65.862500} &   \num{0.150000} &  \num{66.960000} &  \num{15433.250000} \\
std   &  \num{0.057735} &   \num{1.787109} &   \num{0.057735} &   \num{2.302477} &   \num{6984.829579} \\
min   &  \num{0.100000} &  \num{64.260000} &   \num{0.100000} &  \num{64.850000} &   \num{7697.000000} \\
25\%   &  \num{0.100000} &  \num{64.605000} &   \num{0.100000} &  \num{65.457500} &  \num{10628.000000} \\
50\%   &  \num{0.150000} &  \num{65.485000} &   \num{0.150000} &  \num{66.460000} &  \num{15620.500000} \\
75\%   &  \num{0.200000} &  \num{66.742500} &   \num{0.200000} &  \num{67.962500} &  \num{20425.750000} \\
max   &  \num{0.200000} &  \num{68.220000} &   \num{0.200000} &  \num{70.070000} &  \num{22795.000000} \\
\bottomrule
\end{tabular}

\end{table}

To keep consistency with the choice of the best model for the experiment, ``run-20180119152007-rev-l2-u512-clstm-lr10-bw3-atluong'' is chosen based on the test perplexity even if its BLEU score is interesting.

\subsection{Experiment 6 - GPU performance test}
In the previous chapter, Table~\ref{tab:paperspace-flavors} showed that the ``\textit{V100}'' flavor costs more than 6 times the flavor used for experiments 1 to 5 for a theoretical gain in performance of around 43 times (2.6 teraFLOPS against 112 teraFLOPS). This experiment compares the training time for an exact same model on both flavors to bring the needed information for decision-makers.

Table~\ref{tab:run06-results} illustrates the performance results of the same model trained on different machines. The time gain to use machine that costs 6 times more is around 30\%. Besides the time, the perplexities and the BLEU score are similar.
Even though it is only one measure, this experiment aims at raising awareness on the importance to know very well the model trained to be able to choose correctly the best infrastructure with a limited budget.

\begin{table}
    \centering
    \caption[Experiment 6 GPUs performances results]{Experiment 6 GPUs performances results. V100 is 30\% faster than GPU+ while being 600\% more expensive.}
    \label{tab:run06-results}
    \begin{tabular}{crrrrr}
        \toprule
        \tabhead{Flavor} & \tabhead{dev\_ppl} & \tabhead{dev\_bleu} & \tabhead{test\_ppl} & \tabhead{test\_bleu} & \tabhead{Time [s]}\\
        \midrule
        GPU+ & \num{104.71} & \num{0.5} & \num{83.84} & \num{0.3} & \num{2317}\\
        V100 & \num{105.12} & \num{0.3} & \num{84.23} & \num{0.2} & \num{1596}\\
        \bottomrule
    \end{tabular}
\end{table}

\section{Inference}
\label{sec:inference}
This section illustrates how the chatbots actually converse. Since experiment 6 was just a GPU performance test, only the best models of the five other experiments are used. There are 10 different questions or situations, quite simple, and the chatbots' answers are illustrated in Table~\ref{tab:res-inference}.

First, almost all models answers ``Hello !'' with a ``Hi .'', and all models answer ``No .'' to the question ``Are you a follower or a leader'' and ``What ?'' to the affirmation ``Done .''. This shows that NMT is able to learn some basics independently from the models' parameters.
However, the attention mechanism seems to add more sense to the answers. Typically, in the question ``How are you ?'', both attention-based models answer that they are fine whereas other models do not understand the question.

Secondly, the vocabulary plays an important role in the decoding process and it is witness with all the ``<unk>'' tokens in questions such as ``What's your name'' or ``Where are you from''. Personal question learned from more than 9k movie characters is indeed difficult and there are high chances that some words in these particular questions appear less than 5 times (which is the threshold to keep or not a word in the vocabulary).

Finally, even if the models presented in this section have respectively the best test perplexity in their experiments, the answers are still not very convincing.

% \begin{table}
% \begin{longtable}{p{.3\textwidth} p{.1\textheight} p{.2\textheight} p{.35\textheight}}
\begin{longtable}{r|cl}

    \caption[Inference results from best experiments' models]{Inference results from best experiments' models. For each experiment, the best model is picked and is given the same 10 questions or situations. Models used by experiment: (1)~run-20171212014119-norev-l2-u128-clstm-lr10-bw0-voca (2)~run-20171212014119-norev-l2-u256-clstm-lr10-bw0-voc5 (3)~run-20171212014119-rev-l2-u256-clstm-lr10-bw0-voc5 (4)~run-20171212014119-rev-l2-u512-clstm-lr10-bw0-atluong (5)~run-20180119152007-rev-l2-u512-clstm-lr10-bw3-atluong.}
    \label{tab:res-inference}\\
    % \begin{tabular}{l|ll}

    \toprule
    \tabhead{Input} & \tabhead{Experiment} & \tabhead{Output} \\
    \midrule
    \endfirsthead
    \toprule
    \tabhead{Input} & \tabhead{Experiment} & \tabhead{Output} \\
    \midrule
    \endhead
      \midrule
      \multicolumn{3}{r}{\textit{Continued on next page}} \\ % <=======================================
      \bottomrule
    \endfoot % <=======================================
      \bottomrule
    \endlastfoot % <=======================================

    \multirow{5}{*}{Hello !} & 1 & Hi .\\
    & 2 & Hi .\\
    & 3 & Hi .\\
    & 4 & Hi . Hi .\\
    & 5 & I know you are .\\

    \hline
    \multirow{5}{*}{How are you ?} & 1 & What ?\\
    & 2 & I'm not sure .\\
    & 3 & I'm not sure .\\
    & 4 & I'm fine .\\
    & 5 & Fine .\\

    \hline
    \multirow{5}{*}{What's your name ?} & 1 & The man .\\
    & 2 & <unk> .\\
    & 3 & <unk> .\\
    & 4 & Hi .\\
    & 5 & <unk> .\\

    \hline
    \multirow{5}{*}{Where are you from ?} & 1 & The man .\\
    & 2 & <unk> .\\
    & 3 & <unk> .\\
    & 4 & To be .\\
    & 5 & I don't know .\\

    \hline
    \multirow{5}{*}{My name is Rudolf . What is my name ?} & 1 & The man .\\
    & 2 & <unk> .\\
    & 3 & <unk> .\\
    & 4 & Hi . Hi . Hi . Hi . How's business ?\\
    & 5 & <unk> .\\

    \hline
    \multirow{5}{*}{What is your job ?} & 1 &  What ?\\
    & 2 & <unk> .\\
    & 3 & I don't know .\\
    & 4 & My name is John .\\
    & 5 & I don't know .\\

    \hline
    \multirow{5}{*}{Are you a follower or a leader ?} & 1 & No .\\
    & 2 & No .\\
    & 3 & No .\\
    & 4 & No .\\
    & 5 & No .\\

    \hline
    \multirow{5}{*}{What is the color of a yellow car ?} & 1 & You know what ?\\
    & 2 & You know what ?\\
    & 3 & You know what ?\\
    & 4 & It's not a matter of time .\\
    & 5 & I don't know .\\

    \hline
    \multirow{5}{*}{Done .} & 1 &  What ?\\
    & 2 & What ?\\
    & 3 & What ?\\
    & 4 & What ? \\
    & 5 & What ?\\

    \hline
    \multirow{5}{*}{This car was driving too fast !} & 1 &  What ?\\
    & 2 & You were right .\\
    & 3 & What ?\\
    & 4 & What ? \\
    & 5 & I know .\\

    % \end{tabular}
% \end{table}
\end{longtable}


\section{Human evaluation}
In Section~\ref{sec:inference}, having based the best model selection on the test perplexity did not lead to models capable of real conversations. Thus, this section is aiming at showing the best models, not based on the perplexity, but on the perception of comprehension of a Human. In other terms, the chatbot is chosen only if its answers make sense.

There are two models having quite good conversation skills amongst all the trained models, namely ``run-20171212014119-norev-l2-u256-clstm-lr10-bw3-voca'' and ``run-20171212014119-rev-l2-u512-clstm-lr10-bw0-atluong-voc5''. Table~\ref{tab:res-inference-human} shows the conversation skills of both models on the same inference set than in the previous section.

Besides all expectations, the most ``human'' model has neither a limited vocabulary nor reverse input sentence or even attention mechanism. Thus, the perplexity might mean that the optimization is closer to its minima, but it does not mean that the chatbot does its job correctly.

\begin{table}

    \caption[Best models inference from Human evaluation]{Inference results from best models from Human evaluation. (1)~run-20171212014119-norev-l2-u256-clstm-lr10-bw3-voca (2)~run-20171212014119-rev-l2-u512-clstm-lr10-bw0-atluong-voc5.}
    \label{tab:res-inference-human}
    \begin{tabular}{r|cl}

        \toprule
        \tabhead{Input} & \tabhead{Experiment} & \tabhead{Output} \\
        \midrule

        \multirow{2}{*}{Hello !} & 1 & Hi .\\
        & 2 & Hi . Hi .\\

        \hline
        \multirow{2}{*}{How are you ?} & 1 & Fine .\\
        & 2 & I'm fine .\\

        \hline
        \multirow{2}{*}{What's your name ?} & 1 & Paul .\\
        & 2 & Hi .\\

        \hline
        \multirow{2}{*}{Where are you from ?} & 1 & I don't know .\\
        & 2 & To be .\\

        \hline
        \multirow{2}{*}{My name is Rudolf . What is my name ?} & 1 & Captain .\\
        & 2 & Hi . Hi . Hi . Hi . How's business .\\

        \hline
        \multirow{2}{*}{What is your job ?} & 1 &  What ?\\
        & 2 & My name is John .\\

        \hline
        \multirow{2}{*}{Are you a follower or a leader ?} & 1 & No .\\
        & 2 & No .\\

        \hline
        \multirow{2}{*}{What is the color of a yellow car ?} & 1 & What ?\\
        & 2 & It's not a matter of time .\\

        \hline
        \multirow{2}{*}{Done .} & 1 &  What ?\\
        & 2 & What ?\\

        \hline
        \multirow{2}{*}{This car was driving too fast !} & 1 &  What ?\\
        & 2 & What ? \\

        \bottomrule

    \end{tabular}
\end{table}

% interesting inferences:
% \begin{itemize}
%     \item tm_model-20171212014119-l2-u64-clstm-lr10-bw0.infer
%     \item tm_model-20171212014119-l2-u128-cgru-lr10-bw0.infer
%     \item tm_model-20171212014119-l2-u256-cgru-lr10-bw0.infer +1
%     \item tm_model-20171212014119-l2-u256-clstm-lr10-bw0.infer +1
%     \item tm_model-20171212014119-l2-u256-clstm-lr10-bw3.infer +2
%     \item tm_model-20171212014119-l4-u128-cgru-lr10-bw0.infer +0.5 (long sentence)
%     \item tm_model-20171212014119-norev-l2-u256-clstm-lr10-bw0.infer +1 (countries and names are out of the voc)
%     \item tm_model-20171212014119-rev-l2-u512-clstm-lr10-bw0-atluong.infer +2
%     \item tm_model-20171212014119-rev-l2-u512-clstm-lr10-bw0-atscaledluong.infer
%     \item tm_model-20180119152007-rev-l2-u512-clstm-lr10-bw0-atluong.infer +1 (name out of the voc)
% \end{itemize}

% Best parameters :
% LR 1.0
% BW 0 (greedy search best)
% C LSTM
% L 2
% U 512
% voc5
% rev
%
% Run todo:
% - at (luong / scaled_luong), bw (0 / 3), u (256 / 384 / 512) => 12 modeles

% !TEX root = ../main.tex
% Chapter Discussion

\chapter{Discussion} % Main chapter title

\label{Chapter5} % Change X to a consecutive number; for referencing this chapter elsewhere, use \ref{ChapterX}

Other types of seq2seq model that can be exploit as further work for conversational agents
CNN seq2seq \cite{cnn-seq2seq-1705.03122}
GNMT \cite{gnmt-1609.08144}


%----------------------------------------------------------------------------------------
%	THESIS CONTENT - APPENDICES
%----------------------------------------------------------------------------------------

\appendix % Cue to tell LaTeX that the following "chapters" are Appendices

% Include the appendices of the thesis as separate files from the Appendices folder
% Uncomment the lines as you write the Appendices

\include{Appendices/AppendixA}
% !TEX root = ../main.tex
% Appendix A

\chapter{Detailed models and results} % Main appendix title

\label{AppendixB} % For referencing this appendix elsewhere, use \ref{AppendixA}

In this appendix, all of the models and results are presented in a raw table. Chapter~\ref{Chapter4} only illustrates interesting results and analyzes them.

\begin{landscape}

\rowcolors{2}{gray!50}{white}
\begin{longtable}{p{.47\textheight} p{.08\textheight} p{.08\textheight} p{.08\textheight} p{.08\textheight} p{.08\textheight}}
    \hiderowcolors
% \begin{longtable}{p{.3\textheight} | p{.1\textheight} | p{.25\textheight} | p{.3\textheight}}
    % \centering
    % \begin{tabularx}{\textheight}{l | l | X | X}
    \caption[Detailed models and results]{Detailed models and results.}
    \label{tab:apx:res-all}\\

    \toprule
    \tabhead{Model name} & \tabhead{dev\_bleu} & \tabhead{dev\_ppl} & \tabhead{test\_bleu} & \tabhead{test\_ppl} & \tabhead{Time [s]} \\
    \midrule
    \endfirsthead % <=======================================
      \toprule
    \tabhead{Model name} & \tabhead{dev\_bleu} & \tabhead{dev\_ppl} & \tabhead{test\_bleu} & \tabhead{test\_ppl}  & \tabhead{Time [s]}\\
      \midrule
    \endhead % <=======================================
      \midrule
      \multicolumn{6}{r}{\textit{Continued on next page}} \\ % <=======================================
      \bottomrule
    \endfoot % <=======================================
      \bottomrule
    \endlastfoot % <=======================================
    \showrowcolors
    run-20171212014119-norev-l2-u128-cgru-lr01-bw0-voca & \num{0.1} & \num{147.18} & \num{0.0} & \num{118.86} & \num{2398.0}\\
    run-20171212014119-norev-l2-u128-cgru-lr01-bw3-voca & \num{0.0} & \num{147.92} & \num{0.0} & \num{120.06} & \num{3726.0}\\
    run-20171212014119-norev-l2-u128-cgru-lr10-bw0-voca & \num{0.0} & \num{93.65} & \num{0.0} & \num{74.71} & \num{2493.0}\\
    run-20171212014119-norev-l2-u128-cgru-lr10-bw3-voca & \num{0.1} & \num{93.31} & \num{0.0} & \num{73.38} & \num{3450.0}\\
    run-20171212014119-norev-l2-u128-clstm-lr01-bw0-voc5 & \num{0.0} & \num{100.01} & \num{0.0} & \num{80.72} & \num{1609.0}\\
    run-20171212014119-norev-l2-u128-clstm-lr01-bw0-voca & \num{0.1} & \num{146.9} & \num{0.1} & \num{119.07} & \num{2423.0}\\
    run-20171212014119-norev-l2-u128-clstm-lr01-bw3-voc5 & \num{0.0} & \num{99.21} & \num{0.0} & \num{80.41} & \num{1790.0}\\
    run-20171212014119-norev-l2-u128-clstm-lr01-bw3-voca & \num{0.0} & \num{147.97} & \num{0.0} & \num{119.35} & \num{3747.0}\\
    run-20171212014119-norev-l2-u128-clstm-lr10-bw0-voc5 & \num{0.1} & \num{68.04} & \num{0.1} & \num{53.97} & \num{2206.0}\\
    run-20171212014119-norev-l2-u128-clstm-lr10-bw0-voca & \num{0.1} & \num{94.54} & \num{0.2} & \num{75.34} & \num{2471.0}\\
    run-20171212014119-norev-l2-u128-clstm-lr10-bw3-voc5 & \num{0.0} & \num{68.27} & \num{0.0} & \num{54.45} & \num{2517.0}\\
    run-20171212014119-norev-l2-u128-clstm-lr10-bw3-voca & \num{0.0} & \num{93.92} & \num{0.0} & \num{73.86} & \num{3818.0}\\
    run-20171212014119-norev-l2-u256-cgru-lr01-bw0-voca & \num{0.0} & \num{133.85} & \num{0.0} & \num{107.34} & \num{2968.0}\\
    run-20171212014119-norev-l2-u256-cgru-lr01-bw3-voca & \num{0.0} & \num{132.74} & \num{0.0} & \num{107.0} & \num{4131.0}\\
    run-20171212014119-norev-l2-u256-cgru-lr10-bw0-voca & \num{0.1} & \num{86.73} & \num{0.0} & \num{67.68} & \num{2899.0}\\
    run-20171212014119-norev-l2-u256-cgru-lr10-bw3-voca & \num{0.1} & \num{88.91} & \num{0.0} & \num{69.53} & \num{4231.0}\\
    run-20171212014119-norev-l2-u256-clstm-lr01-bw0-voc5 & \num{0.0} & \num{89.55} & \num{0.0} & \num{71.54} & \num{1713.0}\\
    run-20171212014119-norev-l2-u256-clstm-lr01-bw0-voca & \num{0.0} & \num{133.26} & \num{0.0} & \num{107.54} & \num{2914.0}\\
    run-20171212014119-norev-l2-u256-clstm-lr01-bw3-voc5 & \num{0.0} & \num{89.07} & \num{0.0} & \num{71.54} & \num{2044.0}\\
    run-20171212014119-norev-l2-u256-clstm-lr01-bw3-voca & \num{0.0} & \num{133.06} & \num{0.0} & \num{106.48} & \num{4294.0}\\
    run-20171212014119-norev-l2-u256-clstm-lr10-bw0-voc5 & \num{0.1} & \num{64.66} & \num{0.0} & \num{50.62} & \num{1721.0}\\
    run-20171212014119-norev-l2-u256-clstm-lr10-bw0-voca & \num{0.1} & \num{86.88} & \num{0.0} & \num{68.57} & \num{3125.0}\\
    run-20171212014119-norev-l2-u256-clstm-lr10-bw3-voc5 & \num{0.0} & \num{64.41} & \num{0.0} & \num{51.19} & \num{2106.0}\\
    run-20171212014119-norev-l2-u256-clstm-lr10-bw3-voca & \num{0.0} & \num{86.71} & \num{0.0} & \num{68.89} & \num{4669.0}\\
    run-20171212014119-norev-l2-u64-cgru-lr01-bw0-voca & \num{0.4} & \num{172.04} & \num{0.2} & \num{141.75} & \num{2316.0}\\
    run-20171212014119-norev-l2-u64-cgru-lr01-bw3-voca & \num{0.0} & \num{171.55} & \num{0.0} & \num{141.94} & \num{3257.0}\\
    run-20171212014119-norev-l2-u64-cgru-lr10-bw0-voca & \num{0.0} & \num{106.66} & \num{0.0} & \num{85.66} & \num{2135.0}\\
    run-20171212014119-norev-l2-u64-cgru-lr10-bw3-voca & \num{0.0} & \num{106.5} & \num{0.0} & \num{85.5} & \num{3160.0}\\
    run-20171212014119-norev-l2-u64-clstm-lr01-bw0-voc5 & \num{0.3} & \num{115.1} & \num{0.2} & \num{94.6} & \num{2456.0}\\
    run-20171212014119-norev-l2-u64-clstm-lr01-bw0-voca & \num{0.4} & \num{172.0} & \num{0.2} & \num{141.76} & \num{2267.0}\\
    run-20171212014119-norev-l2-u64-clstm-lr01-bw3-voc5 & \num{0.0} & \num{114.71} & \num{0.0} & \num{94.08} & \num{2327.0}\\
    run-20171212014119-norev-l2-u64-clstm-lr01-bw3-voca & \num{0.0} & \num{172.06} & \num{0.0} & \num{142.86} & \num{3556.0}\\
    run-20171212014119-norev-l2-u64-clstm-lr10-bw0-voc5 & \num{0.1} & \num{76.53} & \num{0.0} & \num{61.2} & \num{2500.0}\\
    run-20171212014119-norev-l2-u64-clstm-lr10-bw0-voca & \num{0.1} & \num{106.26} & \num{0.0} & \num{84.76} & \num{2982.0}\\
    run-20171212014119-norev-l2-u64-clstm-lr10-bw3-voc5 & \num{0.0} & \num{76.57} & \num{0.0} & \num{61.46} & \num{3097.0}\\
    run-20171212014119-norev-l2-u64-clstm-lr10-bw3-voca & \num{0.0} & \num{104.92} & \num{0.0} & \num{83.92} & \num{3660.0}\\
    run-20171212014119-norev-l4-u128-cgru-lr01-bw0-voca & \num{0.0} & \num{458.52} & \num{0.0} & \num{406.03} & \num{3934.0}\\
    run-20171212014119-norev-l4-u128-cgru-lr01-bw3-voca & \num{0.0} & \num{458.63} & \num{0.0} & \num{406.0} & \num{4131.0}\\
    run-20171212014119-norev-l4-u128-cgru-lr10-bw0-voca & \num{0.0} & \num{106.34} & \num{0.0} & \num{84.0} & \num{3357.0}\\
    run-20171212014119-norev-l4-u128-cgru-lr10-bw3-voca & \num{0.0} & \num{105.39} & \num{0.0} & \num{84.0} & \num{4026.0}\\
    run-20171212014119-norev-l4-u128-clstm-lr01-bw0-voc5 & \num{0.1} & \num{180.54} & \num{0.0} & \num{161.58} & \num{2429.0}\\
    run-20171212014119-norev-l4-u128-clstm-lr01-bw0-voca & \num{0.0} & \num{458.61} & \num{0.0} & \num{405.97} & \num{3543.0}\\
    run-20171212014119-norev-l4-u128-clstm-lr01-bw3-voc5 & \num{0.0} & \num{179.77} & \num{0.0} & \num{161.88} & \num{2782.0}\\
    run-20171212014119-norev-l4-u128-clstm-lr01-bw3-voca & \num{0.0} & \num{458.63} & \num{0.0} & \num{405.99} & \num{4098.0}\\
    run-20171212014119-norev-l4-u128-clstm-lr10-bw0-voc5 & \num{0.0} & \num{79.73} & \num{0.0} & \num{63.79} & \num{2326.0}\\
    run-20171212014119-norev-l4-u128-clstm-lr10-bw0-voca & \num{0.0} & \num{107.8} & \num{0.0} & \num{84.91} & \num{3439.0}\\
    run-20171212014119-norev-l4-u128-clstm-lr10-bw3-voc5 & \num{0.0} & \num{78.21} & \num{0.0} & \num{62.24} & \num{2623.0}\\
    run-20171212014119-norev-l4-u128-clstm-lr10-bw3-voca & \num{0.0} & \num{107.76} & \num{0.0} & \num{85.08} & \num{4497.0}\\
    run-20171212014119-norev-l4-u256-cgru-lr01-bw0-voca & \num{0.0} & \num{197.4} & \num{0.0} & \num{164.39} & \num{4278.0}\\
    run-20171212014119-norev-l4-u256-cgru-lr01-bw3-voca & \num{0.0} & \num{198.91} & \num{0.0} & \num{166.57} & \num{4803.0}\\
    run-20171212014119-norev-l4-u256-cgru-lr10-bw0-voca & \num{0.0} & \num{102.35} & \num{0.0} & \num{81.33} & \num{3513.0}\\
    run-20171212014119-norev-l4-u256-cgru-lr10-bw3-voca & \num{0.0} & \num{101.42} & \num{0.0} & \num{79.68} & \num{5153.0}\\
    run-20171212014119-norev-l4-u256-clstm-lr01-bw0-voc5 & \num{0.4} & \num{121.43} & \num{0.0} & \num{98.91} & \num{2558.0}\\
    run-20171212014119-norev-l4-u256-clstm-lr01-bw0-voca & \num{0.0} & \num{197.56} & \num{0.0} & \num{165.48} & \num{4145.0}\\
    run-20171212014119-norev-l4-u256-clstm-lr01-bw3-voc5 & \num{0.0} & \num{120.2} & \num{0.0} & \num{97.35} & \num{2776.0}\\
    run-20171212014119-norev-l4-u256-clstm-lr01-bw3-voca & \num{0.0} & \num{198.75} & \num{0.0} & \num{166.05} & \num{4588.0}\\
    run-20171212014119-norev-l4-u256-clstm-lr10-bw0-voc5 & \num{0.0} & \num{74.07} & \num{0.0} & \num{58.53} & \num{2615.0}\\
    run-20171212014119-norev-l4-u256-clstm-lr10-bw0-voca & \num{0.1} & \num{100.39} & \num{0.0} & \num{77.58} & \num{4278.0}\\
    run-20171212014119-norev-l4-u256-clstm-lr10-bw3-voc5 & \num{0.0} & \num{73.37} & \num{0.0} & \num{57.68} & \num{2708.0}\\
    run-20171212014119-norev-l4-u256-clstm-lr10-bw3-voca & \num{0.0} & \num{104.96} & \num{0.0} & \num{83.3} & \num{5931.0}\\
    run-20171212014119-norev-l4-u64-cgru-lr01-bw0-voca & \num{0.0} & \num{462.14} & \num{0.0} & \num{409.46} & \num{3357.0}\\
    run-20171212014119-norev-l4-u64-cgru-lr01-bw3-voca & \num{0.0} & \num{462.34} & \num{0.0} & \num{409.71} & \num{3615.0}\\
    run-20171212014119-norev-l4-u64-cgru-lr10-bw0-voca & \num{0.0} & \num{127.29} & \num{0.0} & \num{101.92} & \num{3027.0}\\
    run-20171212014119-norev-l4-u64-cgru-lr10-bw3-voca & \num{0.0} & \num{124.78} & \num{0.0} & \num{99.42} & \num{3712.0}\\
    run-20171212014119-norev-l4-u64-clstm-lr01-bw0-voc5 & \num{0.0} & \num{335.16} & \num{0.0} & \num{299.35} & \num{2385.0}\\
    run-20171212014119-norev-l4-u64-clstm-lr01-bw0-voca & \num{0.0} & \num{462.41} & \num{0.0} & \num{409.8} & \num{4452.0}\\
    run-20171212014119-norev-l4-u64-clstm-lr01-bw3-voc5 & \num{0.0} & \num{335.16} & \num{0.0} & \num{299.32} & \num{2589.0}\\
    run-20171212014119-norev-l4-u64-clstm-lr01-bw3-voca & \num{0.0} & \num{462.41} & \num{0.0} & \num{409.74} & \num{4351.0}\\
    run-20171212014119-norev-l4-u64-clstm-lr10-bw0-voc5 & \num{0.0} & \num{90.18} & \num{0.0} & \num{72.61} & \num{2275.0}\\
    run-20171212014119-norev-l4-u64-clstm-lr10-bw0-voca & \num{0.0} & \num{123.35} & \num{0.0} & \num{97.98} & \num{3811.0}\\
    run-20171212014119-norev-l4-u64-clstm-lr10-bw3-voc5 & \num{0.0} & \num{89.37} & \num{0.0} & \num{71.91} & \num{2368.0}\\
    run-20171212014119-norev-l4-u64-clstm-lr10-bw3-voca & \num{0.0} & \num{130.95} & \num{0.0} & \num{103.84} & \num{3704.0}\\
    run-20171212014119-norev-l8-u128-cgru-lr01-bw0-voca & \num{0.0} & \num{458.71} & \num{0.0} & \num{406.27} & \num{5548.0}\\
    run-20171212014119-norev-l8-u128-cgru-lr01-bw3-voca & \num{0.0} & \num{458.74} & \num{0.0} & \num{406.27} & \num{6152.0}\\
    run-20171212014119-norev-l8-u128-cgru-lr10-bw0-voca & \num{0.0} & \num{456.72} & \num{0.0} & \num{406.36} & \num{5767.0}\\
    run-20171212014119-norev-l8-u128-cgru-lr10-bw3-voca & \num{0.0} & \num{457.35} & \num{0.0} & \num{406.62} & \num{6063.0}\\
    run-20171212014119-norev-l8-u128-clstm-lr01-bw0-voc5 & \num{0.0} & \num{334.18} & \num{0.0} & \num{299.02} & \num{4846.0}\\
    run-20171212014119-norev-l8-u128-clstm-lr01-bw0-voca & \num{0.0} & \num{458.78} & \num{0.0} & \num{406.3} & \num{5451.0}\\
    run-20171212014119-norev-l8-u128-clstm-lr01-bw3-voc5 & \num{0.0} & \num{334.15} & \num{0.0} & \num{298.97} & \num{4664.0}\\
    run-20171212014119-norev-l8-u128-clstm-lr01-bw3-voca & \num{0.0} & \num{458.64} & \num{0.0} & \num{406.32} & \num{6731.0}\\
    run-20171212014119-norev-l8-u128-clstm-lr10-bw0-voc5 & \num{0.0} & \num{334.44} & \num{0.0} & \num{299.92} & \num{5077.0}\\
    run-20171212014119-norev-l8-u128-clstm-lr10-bw0-voca & \num{0.0} & \num{457.44} & \num{0.0} & \num{406.91} & \num{4871.0}\\
    run-20171212014119-norev-l8-u128-clstm-lr10-bw3-voc5 & \num{0.0} & \num{334.43} & \num{0.0} & \num{299.77} & \num{4841.0}\\
    run-20171212014119-norev-l8-u128-clstm-lr10-bw3-voca & \num{0.0} & \num{456.67} & \num{0.0} & \num{406.92} & \num{6000.0}\\
    run-20171212014119-norev-l8-u256-cgru-lr01-bw0-voca & \num{0.0} & \num{458.78} & \num{0.0} & \num{407.19} & \num{5572.0}\\
    run-20171212014119-norev-l8-u256-cgru-lr01-bw3-voca & \num{0.0} & \num{458.8} & \num{0.0} & \num{406.6} & \num{6096.0}\\
    run-20171212014119-norev-l8-u256-cgru-lr10-bw0-voca & \num{0.0} & \num{457.04} & \num{0.0} & \num{406.69} & \num{5842.0}\\
    run-20171212014119-norev-l8-u256-cgru-lr10-bw3-voca & \num{0.0} & \num{458.75} & \num{0.0} & \num{408.46} & \num{5818.0}\\
    run-20171212014119-norev-l8-u256-clstm-lr01-bw0-voc5 & \num{0.0} & \num{335.57} & \num{0.0} & \num{301.05} & \num{4102.0}\\
    run-20171212014119-norev-l8-u256-clstm-lr01-bw0-voca & \num{0.0} & \num{458.76} & \num{0.0} & \num{406.66} & \num{5654.0}\\
    run-20171212014119-norev-l8-u256-clstm-lr01-bw3-voc5 & \num{0.0} & \num{336.31} & \num{0.0} & \num{301.84} & \num{4260.0}\\
    run-20171212014119-norev-l8-u256-clstm-lr01-bw3-voca & \num{0.0} & \num{458.71} & \num{0.0} & \num{406.99} & \num{5925.0}\\
    run-20171212014119-norev-l8-u256-clstm-lr10-bw0-voc5 & \num{0.0} & \num{334.54} & \num{0.0} & \num{300.71} & \num{4623.0}\\
    run-20171212014119-norev-l8-u256-clstm-lr10-bw0-voca & \num{0.0} & \num{458.08} & \num{0.0} & \num{409.77} & \num{5198.0}\\
    run-20171212014119-norev-l8-u256-clstm-lr10-bw3-voc5 & \num{0.0} & \num{335.35} & \num{0.0} & \num{301.19} & \num{4914.0}\\
    run-20171212014119-norev-l8-u256-clstm-lr10-bw3-voca & \num{0.0} & \num{168.3} & \num{0.0} & \num{140.03} & \num{5268.0}\\
    run-20171212014119-norev-l8-u64-cgru-lr01-bw0-voca & \num{0.0} & \num{462.34} & \num{0.0} & \num{409.77} & \num{4439.0}\\
    run-20171212014119-norev-l8-u64-cgru-lr01-bw3-voca & \num{0.0} & \num{462.24} & \num{0.0} & \num{409.67} & \num{5114.0}\\
    run-20171212014119-norev-l8-u64-cgru-lr10-bw0-voca & \num{0.0} & \num{456.06} & \num{0.0} & \num{406.08} & \num{4507.0}\\
    run-20171212014119-norev-l8-u64-cgru-lr10-bw3-voca & \num{0.0} & \num{456.22} & \num{0.0} & \num{406.57} & \num{5110.0}\\
    run-20171212014119-norev-l8-u64-clstm-lr01-bw0-voc5 & \num{0.0} & \num{335.05} & \num{0.0} & \num{299.31} & \num{4769.0}\\
    run-20171212014119-norev-l8-u64-clstm-lr01-bw0-voca & \num{0.0} & \num{462.33} & \num{0.0} & \num{409.81} & \num{4884.0}\\
    run-20171212014119-norev-l8-u64-clstm-lr01-bw3-voc5 & \num{0.0} & \num{335.18} & \num{0.0} & \num{299.3} & \num{4173.0}\\
    run-20171212014119-norev-l8-u64-clstm-lr01-bw3-voca & \num{0.0} & \num{462.12} & \num{0.0} & \num{409.53} & \num{5097.0}\\
    run-20171212014119-norev-l8-u64-clstm-lr10-bw0-voc5 & \num{0.0} & \num{334.17} & \num{0.0} & \num{299.45} & \num{3866.0}\\
    run-20171212014119-norev-l8-u64-clstm-lr10-bw0-voca & \num{0.0} & \num{456.12} & \num{0.0} & \num{406.45} & \num{5094.0}\\
    run-20171212014119-norev-l8-u64-clstm-lr10-bw3-voc5 & \num{0.0} & \num{334.06} & \num{0.0} & \num{299.6} & \num{5204.0}\\
    run-20171212014119-norev-l8-u64-clstm-lr10-bw3-voca & \num{0.0} & \num{456.84} & \num{0.0} & \num{406.38} & \num{5871.0}\\
    run-20171212014119-rev-l2-u128-clstm-lr01-bw0-voc5 & \num{0.0} & \num{99.11} & \num{0.0} & \num{80.48} & \num{2377.0}\\
    run-20171212014119-rev-l2-u128-clstm-lr01-bw3-voc5 & \num{0.0} & \num{99.81} & \num{0.0} & \num{80.01} & \num{2610.0}\\
    run-20171212014119-rev-l2-u128-clstm-lr10-bw0-voc5 & \num{0.1} & \num{69.17} & \num{0.0} & \num{54.72} & \num{2121.0}\\
    run-20171212014119-rev-l2-u128-clstm-lr10-bw3-voc5 & \num{0.0} & \num{69.49} & \num{0.1} & \num{55.74} & \num{3443.0}\\
    run-20171212014119-rev-l2-u256-clstm-lr01-bw0-voc5 & \num{0.0} & \num{89.78} & \num{0.0} & \num{72.03} & \num{2263.0}\\
    run-20171212014119-rev-l2-u256-clstm-lr01-bw3-voc5 & \num{0.0} & \num{90.18} & \num{0.0} & \num{72.44} & \num{2576.0}\\
    run-20171212014119-rev-l2-u256-clstm-lr10-bw0-atluong & \num{0.1} & \num{68.23} & \num{0.0} & \num{54.2} & \num{2772.0}\\
    run-20171212014119-rev-l2-u256-clstm-lr10-bw0-atscaledluong & \num{0.0} & \num{68.62} & \num{0.0} & \num{54.87} & \num{2646.0}\\
    run-20171212014119-rev-l2-u256-clstm-lr10-bw0-voc5 & \num{0.1} & \num{64.14} & \num{0.0} & \num{50.35} & \num{2476.0}\\
    run-20171212014119-rev-l2-u256-clstm-lr10-bw3-atluong & \num{0.0} & \num{68.56} & \num{0.0} & \num{55.27} & \num{2613.0}\\
    run-20171212014119-rev-l2-u256-clstm-lr10-bw3-atscaledluong & \num{0.0} & \num{67.99} & \num{0.0} & \num{54.24} & \num{2831.0}\\
    run-20171212014119-rev-l2-u256-clstm-lr10-bw3-voc5 & \num{0.0} & \num{64.71} & \num{0.0} & \num{50.72} & \num{2919.0}\\
    run-20171212014119-rev-l2-u384-clstm-lr10-bw0-atluong & \num{0.0} & \num{67.19} & \num{0.0} & \num{53.33} & \num{2448.0}\\
    run-20171212014119-rev-l2-u384-clstm-lr10-bw0-atscaledluong & \num{0.1} & \num{66.33} & \num{0.0} & \num{53.34} & \num{2507.0}\\
    run-20171212014119-rev-l2-u384-clstm-lr10-bw3-atluong & \num{0.0} & \num{66.67} & \num{0.0} & \num{53.11} & \num{2690.0}\\
    run-20171212014119-rev-l2-u384-clstm-lr10-bw3-atscaledluong & \num{0.0} & \num{66.07} & \num{0.0} & \num{52.74} & \num{2827.0}\\
    run-20171212014119-rev-l2-u512-clstm-lr10-bw0-atluong & \num{0.1} & \num{67.0} & \num{0.0} & \num{52.56} & \num{2834.0}\\
    run-20171212014119-rev-l2-u512-clstm-lr10-bw0-atscaledluong & \num{0.0} & \num{67.28} & \num{0.0} & \num{53.04} & \num{2808.0}\\
    run-20171212014119-rev-l2-u512-clstm-lr10-bw3-atluong & \num{0.0} & \num{66.89} & \num{0.0} & \num{52.64} & \num{3195.0}\\
    run-20171212014119-rev-l2-u512-clstm-lr10-bw3-atscaledluong & \num{0.0} & \num{67.25} & \num{0.0} & \num{52.77} & \num{3389.0}\\
    run-20171212014119-rev-l2-u64-clstm-lr01-bw0-voc5 & \num{0.3} & \num{114.62} & \num{0.2} & \num{94.32} & \num{2134.0}\\
    run-20171212014119-rev-l2-u64-clstm-lr01-bw3-voc5 & \num{0.0} & \num{115.0} & \num{0.0} & \num{94.47} & \num{2286.0}\\
    run-20171212014119-rev-l2-u64-clstm-lr10-bw0-voc5 & \num{0.1} & \num{76.5} & \num{0.0} & \num{61.02} & \num{2177.0}\\
    run-20171212014119-rev-l2-u64-clstm-lr10-bw3-voc5 & \num{0.1} & \num{76.59} & \num{0.0} & \num{61.86} & \num{2648.0}\\
    run-20171212014119-rev-l4-u128-clstm-lr01-bw0-voc5 & \num{0.0} & \num{183.41} & \num{0.0} & \num{165.34} & \num{3386.0}\\
    run-20171212014119-rev-l4-u128-clstm-lr01-bw3-voc5 & \num{0.0} & \num{177.38} & \num{0.0} & \num{157.42} & \num{3901.0}\\
    run-20171212014119-rev-l4-u128-clstm-lr10-bw0-voc5 & \num{0.1} & \num{77.69} & \num{0.0} & \num{62.17} & \num{3255.0}\\
    run-20171212014119-rev-l4-u128-clstm-lr10-bw3-voc5 & \num{0.0} & \num{78.16} & \num{0.0} & \num{62.53} & \num{3492.0}\\
    run-20171212014119-rev-l4-u256-clstm-lr01-bw0-voc5 & \num{0.3} & \num{123.06} & \num{0.0} & \num{100.36} & \num{2354.0}\\
    run-20171212014119-rev-l4-u256-clstm-lr01-bw3-voc5 & \num{0.0} & \num{119.4} & \num{0.0} & \num{97.64} & \num{2826.0}\\
    run-20171212014119-rev-l4-u256-clstm-lr10-bw0-voc5 & \num{0.1} & \num{73.74} & \num{0.0} & \num{57.55} & \num{3449.0}\\
    run-20171212014119-rev-l4-u256-clstm-lr10-bw3-voc5 & \num{0.0} & \num{72.0} & \num{0.0} & \num{56.65} & \num{3840.0}\\
    run-20171212014119-rev-l4-u64-clstm-lr01-bw0-voc5 & \num{0.0} & \num{335.16} & \num{0.0} & \num{299.29} & \num{3245.0}\\
    run-20171212014119-rev-l4-u64-clstm-lr01-bw3-voc5 & \num{0.0} & \num{335.04} & \num{0.0} & \num{299.24} & \num{3773.0}\\
    run-20171212014119-rev-l4-u64-clstm-lr10-bw0-voc5 & \num{0.1} & \num{88.84} & \num{0.0} & \num{71.31} & \num{3291.0}\\
    run-20171212014119-rev-l4-u64-clstm-lr10-bw3-voc5 & \num{0.0} & \num{88.39} & \num{0.0} & \num{71.08} & \num{3485.0}\\
    run-20171212014119-rev-l8-u128-clstm-lr01-bw0-voc5 & \num{0.0} & \num{334.23} & \num{0.0} & \num{299.04} & \num{4494.0}\\
    run-20171212014119-rev-l8-u128-clstm-lr01-bw3-voc5 & \num{0.0} & \num{334.2} & \num{0.0} & \num{299.02} & \num{5148.0}\\
    run-20171212014119-rev-l8-u128-clstm-lr10-bw0-voc5 & \num{0.0} & \num{335.02} & \num{0.0} & \num{300.56} & \num{4304.0}\\
    run-20171212014119-rev-l8-u128-clstm-lr10-bw3-voc5 & \num{0.0} & \num{334.16} & \num{0.0} & \num{300.23} & \num{4583.0}\\
    run-20171212014119-rev-l8-u256-clstm-lr01-bw0-voc5 & \num{0.0} & \num{336.17} & \num{0.0} & \num{302.2} & \num{4103.0}\\
    run-20171212014119-rev-l8-u256-clstm-lr01-bw3-voc5 & \num{0.0} & \num{335.85} & \num{0.0} & \num{301.13} & \num{4185.0}\\
    run-20171212014119-rev-l8-u256-clstm-lr10-bw0-voc5 & \num{0.0} & \num{176.54} & \num{0.0} & \num{157.34} & \num{4106.0}\\
    run-20171212014119-rev-l8-u256-clstm-lr10-bw3-voc5 & \num{0.0} & \num{342.64} & \num{0.0} & \num{305.07} & \num{4404.0}\\
    run-20171212014119-rev-l8-u64-clstm-lr01-bw0-voc5 & \num{0.0} & \num{335.2} & \num{0.0} & \num{299.36} & \num{5738.0}\\
    run-20171212014119-rev-l8-u64-clstm-lr01-bw3-voc5 & \num{0.0} & \num{335.06} & \num{0.0} & \num{299.29} & \num{4847.0}\\
    run-20171212014119-rev-l8-u64-clstm-lr10-bw0-voc5 & \num{0.0} & \num{334.17} & \num{0.0} & \num{299.23} & \num{3806.0}\\
    run-20171212014119-rev-l8-u64-clstm-lr10-bw3-voc5 & \num{0.0} & \num{334.2} & \num{0.0} & \num{299.49} & \num{5113.0}\\
    run-20180119152007-rev-l2-u512-clstm-lr10-bw0-atluong & \num{0.2} & \num{66.25} & \num{0.2} & \num{67.26} & \num{11605.0}\\
    run-20180119152007-rev-l2-u512-clstm-lr10-bw0-atscaledluong & \num{0.2} & \num{68.22} & \num{0.2} & \num{70.07} & \num{7697.0}\\
    run-20180119152007-rev-l2-u512-clstm-lr10-bw3-atluong & \num{0.1} & \num{64.26} & \num{0.1} & \num{64.85} & \num{19636.0}\\
    run-20180119152007-rev-l2-u512-clstm-lr10-bw3-atscaledluong & \num{0.1} & \num{64.72} & \num{0.1} & \num{65.66} & \num{22795.0}\\

\end{longtable}

\end{landscape}


%----------------------------------------------------------------------------------------
%	BIBLIOGRAPHY
%----------------------------------------------------------------------------------------

\nocite{*}
\printbibliography[heading=bibintoc]

%----------------------------------------------------------------------------------------

\end{document}
